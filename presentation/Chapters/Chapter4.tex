% Chapter 4

\chapter{Solution n° 3} % Main chapter title

\label{Chapter4} % For referencing the chapter elsewhere, use \ref{Chapter4} 

%----------------------------------------------------------------------------------------

\section{Avancement}

Nous avons récemment trouvé un driver imx219.c qui ne contient aucune libraires
associées et qui fonctionne sur une plateforme hummingboard (SOC i.MX6).

Les sources étant sous forme de patch nous allons cette fois directement l’intégrer à notre système.

Le driver est composé de deux patchs, le premier patch comporte tous les éléments proches du kernel on retrouve :

Kconfig : permet d’ajouter l’imx219 au menuconfig, cela va nous permettre de demander la
compilation du driver. Lorsque l’on modifie le menuconfig on récupère un defconfig
que l’on ajoute en a notre recette.

Makefile : permet d’ajouter la compilation de driver (elle sera activée ou non par le defconfig)

imx219.c : driver qui contient l’ensemble des configurations spécifiques à la caméra.

Le deuxième patch permet d’ajouter les configurations utiles pour faire l’interface entre
le driver et l’utilisateur soit le device-tree.

Problèmes rencontrés :

\begin{itemize}
    \item[-] Des libraires système sont inexistantes, nous les avons donc identifiées et ajoutées au système.
    \item[-] Les versions des librairies existantes ne sont pas compatibles avec notre fichier imx219.c.
    Nous utilisons une version de kernel 3.14 ou bien la version nécessaire est bien plus ressente nous
    l’estimons a 4.12. Pour s’approcher de la versions 4.12 nous sommes actuellement entrain d’appliquer
    les patchs sur une version plus ressente soit 4.1.36 (Krogoth sous Yocto).
\end{itemize}

Vous trouverez ici un lien qui contient l’ensemble des patchs appliqués sur la version 3.14 du kernel :

\href{https://github.com/Alanaitali/meta-openrexpicam/tree/develop/gladwistor/recipes-kernel/linux}{https://github.com/Alanaitali/meta-openrexpicam/tree/develop/gladwistor/recipes-kernel/linux}

\section{Source des patchs}

\href{http://git.armlinux.org.uk/cgit/linux-arm.git/commit/?h=csi-v6&id=e3f847cd37b007d55b76282414bfcf13abb8fc9a}{http://git.armlinux.org.uk/cgit/linux-arm.git/commit/?h=csi-v6\&id=e3f847cd37b007d55b76282414bfcf13abb8fc9a}

\href{http://git.armlinux.org.uk/cgit/linux-arm.git/commit/?h=csi-v6&id=4bd8e1231a2e6eca6a65b565176ea9722611c8dd}{http://git.armlinux.org.uk/cgit/linux-arm.git/commit/?h=csi-v6\&id=4bd8e1231a2e6eca6a65b565176ea9722611c8dd}

\section{Ajout de paquets}

Sachant que nous allions avoir besoin d’une couche v4l-utils et gstreamer comme applicatif (user-space),
nous avons commencé par essayer de le compiler. Cette compilation n’a abouti ni par compilation in-tree
et out-of-tree, ni par une recette Yocto.

\clearpage

Par la suite nous avons ajouté certains paquets à notre image via la variable IMAGE\_INSTALL.

\begin{lstlisting}
DESCRIPTION = "Basic image openrexpicam"
LICENSE = "MIT"
    
inherit core-image
    
IMAGE_INSTALL += " \
    gstreamer \
    i2c-tools \
    gstreamer1.0-plugins-imx \
    gst1.0-fsl-plugin \
    v4l-utils \
"
\end{lstlisting}