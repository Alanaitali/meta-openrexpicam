% Chapter 5

\chapter{État de l'art de la récupération du flux vidéo} % Main chapter title

\label{Chapter5} % For referencing the chapter elsewhere, use \ref{Chapter5}

%----------------------------------------------------------------------------------------

\section{Methodologie par la Webcam USB}

Nous nous aguérissons à l’utilisation de Video4Linux-utils. Nous avons trouvé en
ligne un journal expliquant la récupération du flux vidéo d’une webcam USB avec
V4L2-utils sur imx6. Dans un premier temps, nous avons capturé le flux d’une
webcam USB intégrée à un de nos ordinateurs personnels. Nous avons récupéré une
webcam USB lambda afin de tester les commandes sur la carte openrex et essayer
de capturer le retour vidéo. Une fois fonctionnel nous passerons sur le flux
vidéo de la Raspi Cam v2.

% \section{Commandes}
\begin{enumerate}
  \item Récupérer le port qui est connecté la webcam USB. On utilise la commande:
  \textbf{lsusb -t}
\begin{figure}[th]
    \centering
    \includegraphics[width=1\linewidth,trim={0cm 0,6cm 0cm 12,5cm},clip]{webcam1.png}
    \decoRule
    \caption{Port USB de la webcam USB}  \label{fig:webcam1}
\end{figure}

\item La webcam USB est sur le bus I2C n°1 et l’identifiant du port 1.2.
\begin{figure}[th]
    \centering
    \includegraphics[width=1\linewidth,trim={0cm 0,6cm 0cm 14,5cm},clip]{webcam2.png}
    \decoRule
    \caption{Vérification du port et identifiant de la webcam USB}  \label{fig:webcam2}
\end{figure}

\item Écrire le numéro du bus et l’identifiant du port dans les fichiers bind
et unbind. Le fichier bind permette de  « lier » un appareil USB à son driver et
ainsi le rendre visible pour le système tandis que le fichier unbind lui détache
le driver de son appareil USB pour le « cacher » du système.
\begin{figure}[th]
    \centering
    \includegraphics[scale=0.5,trim={0cm 11,5cm 0cm 1,9cm},clip]{webcam3.png}
    \decoRule
    \caption{Liaison entre la webcam USB et son driver}  \label{fig:webcam3}
\end{figure}

\item La webcam USB est bien reconnu par notre système qui affiche sa
référence (ci-dessus)
\begin{figure}[th]
    \centering
    \includegraphics[scale=0.4,trim={0cm 26,1cm 0cm 1,7cm},clip]{webcam4.png}
    \decoRule
    \caption{Capture du flux vidéo de la webcam USB}  \label{fig:webcam4}
\end{figure}

\item Démarrer un flux vidéo sur notre webcam USB. Avec la commande
\textbf{\# gst-launch-1.0 v4l2src device="/dev/video0" ! video/x-raw,w
idth=640,height=480 ! autovideosink}
Nous en sommes persuadés puisque lorsque nous la lançons, le retour de la
commande se fige à
\begin{lstlisting}
New clock GstSystemClock
\end{lstlisting}
Exactement au même endroit que sur notre propre ordinateur avec notre webcam
intégrée. De plus une petite LED verte s’allume au moment de l’envoi de cette
commande
\begin{figure}[th]
    \centering
    \includegraphics[scale=0.2,trim={10cm 2cm 4cm 9,5cm},clip]{webcam5.png}
    \decoRule
    \caption{LED verte allumée lors de l'envoi de la commande}  \label{fig:webcam5}
\end{figure}

\end{enumerate}

\section{Problèmes}

Notre principal problème pour l’instant est de récupérer ce flux vidéo puisqu’il
est impossible de l’afficher directement dans le terminal, pour tester cela nous
essayons de l’afficher sur nos ordinateurs par le réseau WiFi et protocole UDP.

\section{Point d'amélioration}

Dans un premier temps il est nécessaire de mettre l’image système à jour pour
que l’openrex se connecte à un réseau WiFi, qui sera le même que nos ordinateurs
, et dans un second temps de récupérer ce flux vidéo par le réseau WiFi sur nos
ordinateurs en UDP.

Finalement une fois que nous aurons réussi à afficher le flux vidéo de la webcam
USB sur nos ordinateurs nous essayerons d’effectuer le même processus mais pour
la Raspi Cam v2, qui se pilote par le CSI et non l’USB.

\section{Sources}

\href{http://jas-hacks.blogspot.fr/2014/04/imx6-gstreamer-imx-and-usb-webcam.html}
{Lien du blog de la récupération du flux vidéo par la webcam USB \\
http://jas-hacks.blogspot.fr/2014/04/imx6-gstreamer-imx-and-usb-webcam.html}
