% Chapter 2
\chapter{Solution n° 1} % Main chapter title
\label{Chapter2} % For referencing the chapter elsewhere, use \ref{Chapter2}
%----------------------------------------------------------------------------------------

\begin{description}
  \item[Plateforme :] i.MX6
  \item[Version du Kernel :] 3.14
  \item[Source :] \href{https://www.raspberrypi.org/forums/viewtopic.php?f=43&t=162722}
  {https://www.raspberrypi.org/forums/viewtopic.php?f=43\&t=162722}
\end{description}

Pour décorréler les erreurs de compilation des erreurs de manipulation Yocto,
nous  compilons les fichiers depuis la cross-toolchain du SDK généré par Yocto.
Décrit dans \$CC.

\begin{lstlisting}
#génération du sdk
bitbake openrexpicam-base-image -c populate_sdk
# environement du sdk
source /opt/poky/2.0.3/environment-setup-cortexa9hf-vfp-neon-poky-linux-gnueabi
# apercu de CC
$CC imx219.c
\end{lstlisting}

Appel à des librairies contenues dans l’arborescence de compilation de
différentes recettes.

\begin{lstlisting}
BUILDDIR=/home/diag/workspaceThales/Yocto/fsl-community-bsp/build-imx6rex.com
IMX6S_DIR=$BUILDDIR/tmp/work/imx6s_openrex-poky-linux-gnueabi
#libs linux/*.h
DIRLINUX=$IMX6S_DIR/linux-openrex/3.14-r0/git/include
DIRLINUX2=$IMX6S_DIR/u-boot-openrex/v2015.10+gitAUTOINC+7d8ddd7de7-r0/git
/include
DIRLINUX3=$IMX6S_DIR/core-image-minimal/1.0-r0/sdk/image/opt/poky/2.0.3
/sysroots/cortexa9hf-vfp-neon-poky-linux-gnueabi/usr/include
#libs asm/*.h
DIRASM=$BUILDDIR/tmp/work/x86_64-nativesdk-pokysdk-linux
/nativesdk-linux-libc-headers/4.1-r0/linux-4.1/arch/arm/include/
DIRASM2=$IMX6S_DIR/linux-openrex/3.14-r0/build/arch/arm/include/generated
DIRASM3=$IMX6S_DIR/u-boot-openrex/v2015.10+gitAUTOINC+7d8ddd7de7-r0/git
/arch/arm/include/
#nouvelle commande de compilation "out of tree"
$CC imx219.c -I$DIRLINUX -I$DIRASM -I$DIRASM2 -I$DIRLINUX2
-I$DIRLINUX3 -I$DIRASM3
\end{lstlisting}

Inclusion de 6 dossiers de librairies en option puis, nouvelles dépendances
inexistantes dans le dossier contenant l’arborescence de compilation (builddir).
Recherche d’une autre solution grâce au dialogue avec le professeur de Linux
embarqué. Conseil retenu : Compilation dans l’arborescence Yocto (in-tree)
directe. Certains fichiers restent inexistants dans le kernel 3.14 donc,
patch depuis le kernel 4.1.38.
%----------------------------------------------------------------------------------------
