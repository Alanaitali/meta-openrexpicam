%%%%%%%%%%%%%%%%%%%%%%%%%%%%%%%%%%%%%%%%%
% Masters/Doctoral Thesis 
% LaTeX Template
% Version 2.5 (27/8/17)
%
% This template was downloaded from:
% http://www.LaTeXTemplates.com
%
% Version 2.x major modifications by:
% Vel (vel@latextemplates.com)
%
% This template is based on a template by:
% Steve Gunn (http://users.ecs.soton.ac.uk/srg/softwaretools/document/templates/)
% Sunil Patel (http://www.sunilpatel.co.uk/thesis-template/)
%
% Template license:
% CC BY-NC-SA 3.0 (http://creativecommons.org/licenses/by-nc-sa/3.0/)
%
%%%%%%%%%%%%%%%%%%%%%%%%%%%%%%%%%%%%%%%%%

%----------------------------------------------------------------------------------------
%	PACKAGES AND OTHER DOCUMENT CONFIGURATIONS
%----------------------------------------------------------------------------------------

\documentclass[
11pt, % The default document font size, options: 10pt, 11pt, 12pt
%oneside, % Two side (alternating margins) for binding by default, uncomment to switch to one side
french, % ngerman for German
singlespacing, % Single line spacing, alternatives: onehalfspacing or doublespacing
%draft, % Uncomment to enable draft mode (no pictures, no links, overfull hboxes indicated)
%nolistspacing, % If the document is onehalfspacing or doublespacing, uncomment this to set spacing in lists to single
%liststotoc, % Uncomment to add the list of figures/tables/etc to the table of contents
%toctotoc, % Uncomment to add the main table of contents to the table of contents
%parskip, % Uncomment to add space between paragraphs
%nohyperref, % Uncomment to not load the hyperref package
headsepline, % Uncomment to get a line under the header
%chapterinoneline, % Uncomment to place the chapter title next to the number on one line
%consistentlayout, % Uncomment to change the layout of the declaration, abstract and acknowledgements pages to match the default layout
]{MastersDoctoralThesis} % The class file specifying the document structure

\usepackage[utf8]{inputenc} % Required for inputting international characters
\usepackage[T1]{fontenc} % Output font encoding for international characters
\usepackage{xparse}
\usepackage{mathpazo} % Use the Palatino font by default

\usepackage[backend=bibtex,style=authoryear,natbib=true]{biblatex} % Use the bibtex backend with the authoryear citation style (which resembles APA)

\addbibresource{example.bib} % The filename of the bibliography

\usepackage[autostyle=true]{csquotes} % Required to generate language-dependent quotes in the bibliography
\usepackage{enumitem} % package for enumeration with bullet in french
\usepackage{float} % package for the float objects
\usepackage{graphicx}
\usepackage{lscape}
\usepackage{xparse}

\usepackage[most]{tcolorbox}

\tcbset
{
    frame code={}
    center title,
    left=0pt,
    right=0pt,
    top=0pt,
	bottom=0pt,
	colupper=white,
    colback=black,
    colframe=white,
    width=\dimexpr\textwidth\relax,
    enlarge left by=0mm,
    boxsep=3pt,
	arc=0pt,outer arc=0pt,
	halign=left,
}

\ExplSyntaxOn
\NewDocumentCommand{\tabhead}{ O{\bfseries} m }
 {
  \seq_set_split:Nnn \l_tmpa_seq { & } { #2 }
  #1 \seq_use:Nn \l_tmpa_seq { & #1 } \\
 }
\ExplSyntaxOff

\let\cleardoublepage\clearpage

\usepackage{inconsolata}
\usepackage{color}
\usepackage{listings}
\usepackage{textcomp}

%----------------------------------------------------------------------------------------
%	FOR THE LIST OF EQUATIONS
%----------------------------------------------------------------------------------------
\usepackage{tocloft}
\usepackage{xstring}
%\makeatletter

% we use this for our refernces as well
%\AtBeginDocument{\renewcommand{\ref}[1]{\mbox{\autoref{#1}}}}

% redefinition of \equation for convenience
%\let\oldequation = \equation
%\let\endoldequation = \endequation
%\AtBeginDocument{\let\oldlabel = \label}% \AtBeginDocument because hyperref redefines \label
%\newcommand{\mynewlabel}[1]{%
%  \StrBehind{#1}{eq:}[\Str]% remove "eq:" from labels
%  \myequations{\Str}\oldlabel{#1}}
 % \renewenvironment{equation}{%
 % \oldequation
 % \let\label\mynewlabel
%}{\endoldequation}

%\newcommand{\listequationsname}{Liste des équations}
%\newlistof{myequations}{equ}{\listequationsname}
%\newcommand{\myequations}[1]{%
%      \addcontentsline{equ}{myequations}{\protect\numberline{\theequation}#1}}
%	  \setlength{\cftmyequationsindent}{1.5em}
%	  \setlength{\cftmyequationsnumwidth}{2.3em}

%\makeatother

\lstset
{
   	language=bash, %% Troque para PHP, C, Java, etc... bash é o padrão
   	basicstyle=\ttfamily\small,
   	numberstyle=\footnotesize,
   	numbers=left,
   	backgroundcolor=\color{gray!10},
   	frame=single,
   	tabsize=2,
   	rulecolor=\color{black!30},
   	title=\lstname,
    escapeinside={\%*}{*)},
    breaklines=true,
    breakatwhitespace=true,
    %framextopmargin=0pt,
    %framexbottommargin=0pt,
    inputencoding=utf8,
    extendedchars=true,
    literate={á}{{\'a}}1 {ã}{{\~a}}1 {é}{{\'e}}1,
}

%----------------------------------------------------------------------------------------
%	MARGIN SETTINGS
%----------------------------------------------------------------------------------------

\geometry{
	paper=a4paper, % Change to letterpaper for US letter
	inner=2.5cm, % Inner margin
	outer=3.8cm, % Outer margin
	bindingoffset=.5cm, % Binding offset
	top=1.5cm, % Top margin
	bottom=1.5cm, % Bottom margin
	%showframe, % Uncomment to show how the type block is set on the page
}

%----------------------------------------------------------------------------------------
%	THESIS INFORMATION
%----------------------------------------------------------------------------------------

\thesistitle{Intégration du driver IMX219 sur une OpenRex Basic} % Your thesis title, this is used in the title and abstract, print it elsewhere with \ttitle
\supervisor{Patrick \textsc{Piquart} \\
			David \textsc{Coué}} % Your supervisor's name, this is used in the title page, print it elsewhere with \supname
\examiner{} % Your examiner's name, this is not currently used anywhere in the template, print it elsewhere with \examname
\degree{Master Systèmes Embarqués} % Your degree name, this is used in the title page and abstract, print it elsewhere with \degreename
\author{Alan \textsc{Ait-Ali}, Martin \textsc{Laporte}, \\
Clément \textsc{Ailloud} \& Romain \textsc{Petit} \\ } % Your name, this is used in the title page and abstract, print it elsewhere with \authorname
\addresses{} % Your address, this is not currently used anywhere in the template, print it elsewhere with \addressname

\subject{Biological Sciences} % Your subject area, this is not currently used anywhere in the template, print it elsewhere with \subjectname
\keywords{} % Keywords for your thesis, this is not currently used anywhere in the template, print it elsewhere with \keywordnames
\university{YNOV} % Your university's name and URL, this is used in the title page and abstract, print it elsewhere with \univname
\department{Département Aéronautique \& Systèmes Embarqués} % Your department's name and URL, this is used in the title page and abstract, print it elsewhere with \deptname
\group{Département Aéronautique \& Systèmes Embarqués} % Your research group's name and URL, this is used in the title page, print it elsewhere with \groupname
\faculty{YNOV} % Your faculty's name and URL, this is used in the title page and abstract, print it elsewhere with \facname

\AtBeginDocument{
\hypersetup{pdftitle=\ttitle} % Set the PDF's title to your title
\hypersetup{pdfauthor=\authorname} % Set the PDF's author to your name
\hypersetup{pdfkeywords=\keywordnames} % Set the PDF's keywords to your keywords
}

\begin{document}

\frontmatter % Use roman page numbering style (i, ii, iii, iv...) for the pre-content pages

\pagestyle{plain} % Default to the plain heading style until the thesis style is called for the body content

%----------------------------------------------------------------------------------------
%	TITLE PAGE
%----------------------------------------------------------------------------------------

\begin{titlepage}
\begin{center}

\vspace*{.06\textheight}
{\scshape\LARGE \univname\par}\vspace{1.5cm} % University name
\textsc{\Large Projet de Master en partenariat avec Thales}\\[0.5cm] % Thesis type

\HRule \\[0.4cm] % Horizontal line
{\huge \bfseries \ttitle\par}\vspace{0.4cm} % Thesis title
\HRule \\[1.5cm] % Horizontal line
 
\begin{minipage}[t]{0.4\textwidth}
\begin{flushleft} \large
\emph{Auteur :}\\
Alan \textsc{Ait-Ali} \\
Martin \textsc{Laporte} \\
Clément \textsc{Ailloud} \\
Romain \textsc{Petit}
%\authorname % Author name - remove the \href bracket to remove the link
\end{flushleft}
\end{minipage}
\begin{minipage}[t]{0.4\textwidth}
\begin{flushright} \large
\emph{Superviseurs :} \\
\supname % Supervisor name - remove the \href bracket to remove the link  
\end{flushright}
\end{minipage}\\[3cm]

\large \textit{Rapport final présentant le travail effectué sur \\
				l'intégration du driver IMX219 sur une OpenRex Basic}\\[0.3cm] % University requirement text
\textit{au sein du}\\[0.4cm]
\deptname\\[2cm] % Research group name and department name

\includegraphics[scale=0.7]{Figures/Logo_Ynov.png}\\[1cm] % University/department logo - uncomment to place it

{\large \today}\\[4cm] % Date

 
\vfill
\end{center}
\end{titlepage}

%----------------------------------------------------------------------------------------
%	DECLARATION PAGE
%----------------------------------------------------------------------------------------

%\begin{declaration}
%\addchaptertocentry{\authorshipname} % Add the declaration to the table of contents
%\noindent Je, \authorname, declare que ce mémoire intitulé, \enquote{\ttitle} et le travail qui y est présenté, est le mien. Je confirme que :

%\begin{itemize}[label=$\bullet$] 
%\item Ce travail a été réalisé entièrement ou principalement pour l'obtention de mon diplôme au sein de l'école E.S.T.E.I.
%\item Si une partie quelconque de cette thèse a déjà été soumise pour obtenir un diplôme ou toute autre qualification dans cette école ou dans une autre institution, cela est clairement indiqué.
%\item Lorsque je consulte le travail publié d'autrui, cela est toujours clairement attribué.
%\item Lorsque je cite le travail des autres, la source est toujours donnée. À l'exception de ces citations, cette thèse est entièrement mon propre travail.
%\item J'ai pris connaissances de toutes les principales sources d'aide.
%\item Lorsque la thèse est basée sur le travail effectué par moi-même conjointement avec d'autres, j'ai précisé exactement ma contribution et ce qui a été fait par les autres.\\
%\end{itemize}
 
%\noindent Signature:\\
%\rule[0.5em]{25em}{0.5pt} % This prints a line for the signature
 
%\noindent Date:\\
%\rule[0.5em]{25em}{0.5pt} % This prints a line to write the date
%\end{declaration}

%\cleardoublepage

%----------------------------------------------------------------------------------------
%	QUOTATION PAGE
%----------------------------------------------------------------------------------------

%\vspace*{0.2\textheight}

%\noindent\enquote{\itshape Thanks to my solid academic training, today I can write hundreds of words on virtually any topic without possessing a shred of information, which is how I got a good job in journalism.}\bigbreak

%\hfill Dave Barry

%----------------------------------------------------------------------------------------
%	ABSTRACT PAGE
%----------------------------------------------------------------------------------------

\begin{abstract}
\addchaptertocentry{\abstractname} % Add the abstract to the table of contents

Actuellement, les casques de réalité augmentée pour pilote migrent vers le domaine civil. Thales a
réalisé un prototype fonctionnant avec une Raspberry Pi et la Raspberry Pi Camera pour 
l’acquisition d’images ou de flux vidéo. Thales a fait appel aux étudiants du campus Ynov pour 
participer à l’amélioration de ce prototype. 

L’objectif du projet est de réaliser le portage du driver de la caméra sur un autre système 
d’exploitation que Raspberry Pi. Ce système devra donc être capable faire des captures d’image et un
flux vidéo de la caméra depuis une autre carte, l’Openrex-IMX6-Quad car pour des raisons internes à 
l’équipe de développement Thales-LUCY.

Avec Yocto, nous avons généré dans un premier temps, un système d’exploitation fonctionnel sur la 
carte cible (kernel Linux 3.14 puis 4.1). Par la suite, nous avons essayé de faire fonctionner des
codes existants en les adaptant à la carte Openrex. N’y arrivant pas, nous avons alors entrepris la
rédaction d’un driver de la caméra. Actuellement, nous avons réussi à établir une connexion I2C entre
la caméra et l’IMX6 Openrex en utilisant le port CSI-2.

Notre travail étant inachevé, ce rapport permet de partager nos connaissances acquises lors du 
déroulement du projet. Il est nécessaire pour une bonne compréhension d’être introduit à l’usage de 
Yocto. \bigskip

Nowadays augmented reality pilot’s helmet move to the civil domain. At the moment Thales group 
prototyped one working with a Raspberry Pi and it’s associated raspberry camera v2 for picture and
video capture. Thales called ynov campus students for applications at being part of the prototype
improvement. 

Our aim is to build an Raspberry Pi camera, picture and video capture compatible, operating system 
on l’Openrex-IMX6-Quad. Indeed for Thales-LUCY team internal reasons the initial Raspberry Pi board 
won’t be used in the MVP project state.

In a first hand we built a Yocto-project generated target board operating system (kernel Linux 3.14 
then 4.1). In a second hand we tried to port existing source codes to the Openrex board. Unable to 
succeed, we undertook the camera driver redaction from scratch. We already connected the camera 
module with both CSI and I2C communication protocols as the mobile industry processor interface 
(MIPI) standard describes.

As our work in unfinished, the present report hand our state of knowledge and achievement over. 
Linux operating system Yocto generation usage knowledge is needed for it’s good understanding.

%\ldots
\end{abstract}

%----------------------------------------------------------------------------------------
%	ACKNOWLEDGEMENTS
%----------------------------------------------------------------------------------------

\begin{acknowledgements}
\addchaptertocentry{\acknowledgementname} % Add the acknowledgements to the table of contents
En premier lieu, nous tenons à remercier l’équipe de Thales qui a bien voulu nous faire confiance
pour porter un de leur projet et pour la bienveillance dont ils ont fait preuve lorsque les voies
empruntées n’étaient pas les bonnes. \medskip

Nous remercions également l’équipe enseignante de notre formation sans qui ce projet n’aurait pas pu
être, plus particulièrement Monsieur Pierre-Jean TEXIER pour ses conseils et son soutien technique. \medskip

Nous voudrions ensuite remercier Monsieur Patick PIQUART et Monsieur David COUE pour nous avoir permis
de participer au projet au sein du groupe Ynov, ainsi que de leurs appuis en termes de gestion de projet.

\end{acknowledgements}

%----------------------------------------------------------------------------------------
%	LIST OF CONTENTS/FIGURES/TABLES PAGES
%----------------------------------------------------------------------------------------

\tableofcontents % Prints the main table of contents
\clearpage

\listoffigures % Prints the list of figures
\clearpage

\listoftables % Prints the list of tables
\clearpage

%\listofmyequations


%----------------------------------------------------------------------------------------
%	ABBREVIATIONS
%----------------------------------------------------------------------------------------

%\begin{abbreviations}{ll} % Include a list of abbreviations (a table of two columns)

%\textbf{LAH} & \textbf{L}ist \textbf{A}bbreviations \textbf{H}ere\\
%\textbf{WSF} & \textbf{W}hat (it) \textbf{S}tands \textbf{F}or\\

%\end{abbreviations}

%----------------------------------------------------------------------------------------
%	PHYSICAL CONSTANTS/OTHER DEFINITIONS
%----------------------------------------------------------------------------------------

%\begin{constants}{lr@{${}={}$}l} % The list of physical constants is a three column table

% The \SI{}{} command is provided by the siunitx package, see its documentation for instructions on how to use it

%Speed of Light & $c_{0}$ & \SI{2.99792458e8}{\meter\per\second} (exact)\\
%Constant Name & $Symbol$ & $Constant Value$ with units\\

%\end{constants}

%----------------------------------------------------------------------------------------
%	SYMBOLS
%----------------------------------------------------------------------------------------

%\begin{symbols}{lll} % Include a list of Symbols (a three column table)

%$a$ & distance & \si{\meter} \\
%$P$ & power & \si{\watt} (\si{\joule\per\second}) \\
%Symbol & Name & Unit \\

%\addlinespace % Gap to separate the Roman symbols from the Greek

%$\omega$ & angular frequency & \si{\radian} \\

%\end{symbols}

%----------------------------------------------------------------------------------------
%	DEDICATION
%----------------------------------------------------------------------------------------

%\dedicatory{For/Dedicated to/To my\ldots} 

%----------------------------------------------------------------------------------------
%	THESIS CONTENT - CHAPTERS
%----------------------------------------------------------------------------------------

\mainmatter % Begin numeric (1,2,3...) page numbering

\pagestyle{thesis} % Return the page headers back to the "thesis" style

% Include the chapters of the thesis as separate files from the Chapters folder
% Uncomment the lines as you write the chapters

% Chapter 1

\chapter{Organisation de l'équipe et planning} % Main chapter title

\label{Chapter1} % For referencing the chapter elsewhere, use \ref{Chapter1}

%----------------------------------------------------------------------------------------

\section{Organisation de l'équipe}

Afin de se concentrer sur la "Solution n°2" lundi 8 nous nous sommes lancés dans
l'annalyse du code c du driver et de son fonctionnement. Le code réparti
entre chacun, nous avons cherché à comprendre les utilités des structures et de
l'organisation d'un client kernel. Nous avons donc chacuns poursuivit les appels
aux fichiers systemes, en mutualisant les informations oralement.

Grâce à cette organisation agile nous avons pu préciser la source du problème
avant d'en chercher la solution.

Par la suite Romain suivi par Alan se sont lancés dans le second objectif
d'évolution, la compilation d'un noyau en version 4.14. Ce afin de précéder le
portage du driver à ce kernel.

\section{Planning}

Ci-dessous, un planning des tâches effectuées en 2018.

\begin{figure}[th]
    \centering
    \includegraphics[width=1\linewidth]{planning_petit.png}
    \decoRule
    \caption{Avancement général du projet}
    \label{fig:planning}
\end{figure}
\begin{description}
    \item[AL =] Alan Ait-Ali
    \item[MA =] Martin LAPORTE
    \item[RO =]  Romain Petit
    \item[CL =] Clément Ailloud
  \end{description}
  
  La totalité du planning est accessible sur notre git, sur la branche "presentation"
  à l'emplacement "meta-openrexpicam/presentation2/gantthales.planner" sous forme
  de fichier à ouvrir avec le logiciel "planner" (sudo aptitude install planner).

\clearpage
\begin{landscape}
\begin{figure}[th]
    \centering
    \includegraphics[width=1\linewidth]{planning_grand.png}
    \decoRule
    \caption{Avancement détaillé du projet}
    \label{fig:planning}
\end{figure}
\end{landscape}

%----------------------------------------------------------------------------------------

% Chapter 2

\chapter{Gestion de projet} % Main chapter title

\label{Chapter2} % For referencing the chapter elsewhere, use \ref{Chapter2} 

%----------------------------------------------------------------------------------------

\section{Organisation de l'équipe}

\subsection{Membres du projet}

\begin{table}[htp]
    \centering
    \begin{tabular}{|p{0.3\textwidth}|p{0.6\textwidth}|}
        \hline
        \tabhead{Nom / Prénom & Rôle}
        \hline
        Patrick \textsc{Piquart} & Scrum Master : Responsable de l'organisation interne et de la communication avec le client \\ \hline
        David \textsc{Coué} & Product Owner : Responsable de la communication avec le client \\ \hline
        Alan \textsc{Ait-Ali}  & Développeur \\ \hline
        Martin \textsc{Laporte} & Développeur \\ \hline
        Clément \textsc{Ailloud} & Développeur \\ \hline
        Romain \textsc{Petit} & Développeur \\ \hline
    \end{tabular}
    \caption{Membres du projet} \label{tab:membres} 
\end{table}

\subsection{Organisation externe}

À l’initiative de ce projet, il s’est tenu une présentation pour les sections Master 1 et Master 2 de
la formation. À l’issue de celle-ci, M. David Coué et M. Patrick Picard ont formé un groupe de 4
étudiants. Initialement, M. Patrick Picard a pris le rôle de “Scrum Master”, c’est avec lui que s’est
faite la “kick-off review”. Une réunion a ensuite eu lieu pour que le groupe de travail puisse bien
cerner les enjeux et qu’il soit en accord sur les méthodes de suivi de projet. Matérialisé par un
“backlog product”, cette méthode a permis d’avoir une bonne communication interne et de suivre 
l’avancement des différentes tâches. Pour la communication avec le client, le groupe d'étudiant
devait impérativement passer par  le “Product Owner” ou le “Scrum Master” pour faire remonter les
informations. \medskip

Quelques temps après la “kick-off review", le groupe de travail rencontra les ingénieurs en charge
du projet. À l’occasion de cet entretien fut exposé l’état de l’art ainsi que des avis sur certains
points techniques. \medskip

L’emploi du temps du projet et celui du Scrum Master n’étant pas compatible, les “daily review” se
sont tenues quasi-quotidiennement jusqu’à fin décembre entre les 4 étudiants. Face à la désinformation
du “Scrum Master” et des clients, un changement de méthode nous a été conseillé en janvier. La daily
fut alors remplacée par un rapport d’avancement. Cette méthode a permis de renouer le lien entre le
travail fournit par les étudiants et le client. Cela a fait prendre une toute nouvelle direction au
projet, malheureusement un peu tard car le projet devra s'arrêter le 25 février.\medskip

La gestion du partage des tâches au sein de l’équipe s’est auto-organisée selon les tâches à
développer. On ne pourra pas attribuer de rôle bien précis à chacun car en fonction de l'avancée du
projet les rôles se sont intervertis.

\begin{figure}[th]
    \centering
    \includegraphics[scale=0.35]{Figures/fleche.png}
    \decoRule
    \caption{Chronologie du projet}  \label{fig:fleche}
\end{figure}

\subsection{Organisation interne}

À l’initialisation du projet, nous avons décidé de nous séparer en deux groupes. Le
premier groupe portera son étude depuis la couche haut niveau. À l’inverse, le second
groupe partira du bas niveau (modèle OSI). Pour mieux appréhender les différents
problèmes et axes de développement dans leur ensemble. \medskip

Le binôme Clément et Romain recherchaient à faire le lien depuis les couches applicatives
vers le kernel. Le binôme Alan et Martin au contraire cherchaient à établir en priorité les
couches basses pour ensuite les rendre compatibles avec le kernel.
Pour résumer les deux binômes n’avaient pas le même point de départ tout en ayant le
même kernel comme point d’arrivée. Clément et Romain sont donc chargés de
comprendre l’utilisation de Gstreamer et de la couche V4L2 ; Martin et Alan de rendre les
drivers compatibles avec notre système.  \medskip

Afin de se concentrer sur l’option la plus prometteuse (c.f. imx219 - Nvidia-Tegra
Chromium-Os) nous nous sommes lancés dans l’analyse du code C du pilote, donc
de son fonctionnement interne. Le code réparti entre chacun, nous avons cherché à
comprendre les utilités des structures et de l’organisation du code. Nous
avons poursuivi les appels aux fichiers systèmes, en mutualisant les informations
oralement. Grâce à cette organisation agile nous avons pu préciser la source du problème
avant d’en chercher la solution.  \medskip

Par la suite Romain et Alan se sont lancés dans le second objectif d’évolution. Ils ont
compilé un noyau linux récent, afin de précéder le portage des sources spécifiques à la
carte de développement (board support package, BSP) Openrex de la distribution.
En raison du peu de résultats et des nouvelles pistes données par le client, le groupe s’est
orienté vers l’étude d’un driver existant. Chacun est chargé de comprendre et modifier les
fonctions pour les rendre compatibles avec notre camera.  \medskip

En clair, le groupe s'est organisé de façon à progresser le plus rapidement possible sur
une même piste. Au maximum, deux pistes différentes étaient étudiées en parallèle.
Lorsque le travail pouvait être divisé, chaque binôme s’occupait d’en prendre une partie
pour ensuite tout remettre en commun.

\section{Diagramme de Gantt}

\begin{figure}[!htb]
    \centering
    \includegraphics[angle=90,trim={2.5cm 2cm 0cm 3.5cm},clip,scale=0.35]{Figures/gantt.png}
    \decoRule
    \caption{Avancement détaillé du projet} \label{fig:planning}
\end{figure}

\clearpage

\section{Outils utilisés}

\subsection{Versionning des codes}

GitHub est un service de versioning ainsi qu’un service web d’hébergement. Il permet de
stocker toutes les sources d’un projet en différenciant par des versions. En effet, si nous
effectuons des modifications corrompant tout le projet, il est possible de récupérer la
version fonctionnelle si elle a été versionnée. \medskip

GitHub est trés utilisé dans le monde professionnel puisqu’il permet à plusieurs dizaines
de personnes à travailler simultanément sur un projet, par exemple le système
d’exploitation Linux sur GitHub mis à jour par des centaines de contributeurs. \medskip

Une fonctionnalité importante aussi est le système de “branches”. Si l’on souhaite séparer
certaines parties indépendantes d’un projet on utilise ce système. Cela permet de travailler
sur les mêmes fichiers en parallèle sans que les modifications apportées par les autres ne
posent de problèmes, à condition que les lignes concernées soient différentes. Le
versioning d’un projet s’effectue avec les quelques commandes principales ci-dessous.

\subsubsection{Commandes}

\begin{itemize}
    \item[Clone : ] Crée un dépôt local sur l’ordinateur depuis un dépôt en ligne
    \item[Add : ] Ajoute les fichiers ou dossiers dans l’index que nous voulons versionner sur le
    GitHub
    \item[Commit : ] Transfère le contenu de l’index vers le répertoire local ; commet la version. il est
    possible de rajouter un commentaire avec l’option -m
    \item[Push : ] Pousse les fichiers et dossiers contenus dans le répertoire local vers le dépôt en
    ligne après la commande commit
    \item[Pull : ] Actualise la branche locale sur l’ordinateur depuis un dépôt en ligne
    \item[Branch : ] Créé une nouvelle branche
    \item[Checkout ] Change de branche
    \item[Diff : ] Affiche les différences de fichiers entre le contenu local et le contenu du dépôt en
    ligne
\end{itemize}

Voic un shéma résumant graphiquement toutes les commandes ci-dessus :

\begin{figure}[!htb]
    \centering
    \includegraphics[trim={0cm 0cm 0cm 0cm},clip,scale=0.8]{Figures/git.png}
    \decoRule
    \caption{Résumé des commandes git} \label{fig:git}
\end{figure}

Il existe sur internet de nombreuses recommandations pour entretenir un dépôt git propre.
Étant débutants du principe, nous nous sommes concentrés sur l’aspect fonctionnel.

Ci-dessous, l’adresse de notre GitHub :

\href{https://github.com/Alanaitali/meta-openrexpicam}{https://github.com/Alanaitali/meta-openrexpicam}

\subsection{Communication}

A l’aube du projet quand nous avons choisi nos méthodes de communication, nous avons
décidé d'utiliser le logiciel (et hébergeur) de dialogue instantané nommé Slack. D’une part
nous avions déjà défini que nous nous verrions deux jours hebdomadairement d’autre part
nous avions déjà choisi de dialoguer avec l’équipe Thales en passant par notre
superviseur par mail qui relayerai les requêtes. Enfin comme précisé ci-dessus, un git
était à l’œuvre pour les échanges de code.

\subsubsection{Avantages de Slack}

\begin{itemize}
    \item[-] Interface très complète de dialogue en groupe
    \item[-] Capacité de rechercher parmi les messages
    \item[-] Conversations publiques/privées, en sous-groupe... etc
    \item[-] Accessible depuis un navigateur, et application pour ordinateur et smartphone de tous
    types, utile pour être notifié.
    \item[-] Gestions séparées des projets pour éviter de s’éparpiller sur un autre contenu (orientation
    professionnelle)
    \item[-] Appel à des applications en ligne possible (calc, github, google\_drive ...)
\end{itemize}

\subsubsection{Inconvénients de Slack}

\begin{itemize}
    \item[-] Gestion des logins indépendantes entre les projets (contre intuitif)
    \item[-] Applications mobiles et ordinateur non-native et donc plus consommatrices
    \item[-] Interface nouvelle avec des capacités restées inexploitées (planning, fils de discussions
    sur un message)
\end{itemize}

\subsection{Yocto}

OpenEmbedded, était à son commencement en 2003, un projet de la société du même
nom, rejointe plus tard par OpenZaurus. Avant son arrivée, l'outil massivement employé
était Buildroot, celui-ci étant principalement prévu pour construire des systèmes de fichiers
et non des distributions voire des SDK GNU-Linux. \medskip

En 2010 OpenEmbedded devient un “lab workgroup” de la fondation Linux avec 22
entreprises collaborant entre-elles. En 2011, lors de son rachat par Intel, le projet est
nommé Yocto. Celui-ci a pour but de faciliter la conception de systèmes Linux avec une
empreinte mémoire minime et la compilation croisée. Il permet théoriquement de
développer une distribution spécifique aux besoins d’un utilisateur indépendamment de la
cible et du poste de développement. \medskip

\clearpage

Enrichi par un grand nombre d’entreprises tel que NXP ou Texas instrument, le projet
Yocto s’est développé et est maintenant maintenu autour de deux blocs :

\begin{itemize}
    \item[-] Bitbake : outil de construction dérivé du gestionnaire de paquet “portage” par
    l’équipe du projet OpenEmbedded. Bitbake active les différents ingrédients utiles à
    la compilation des recettes.
    \item[-] OpenEmbedded-core : les sources de base sous forme de métadonnées pour la
    génération d’un système GNU/Linux base, la distribution poky.
\end{itemize}

Yocto peut être rendu compatible avec de très nombreuses combinaisons de SBC et SOM
grâce à une gestion open-source et à la méthode de développement incrémental des
recettes qu’il instaure. Il permet de générer une distribution complète (image, bootloader,
SDK, rootfs, device-tree...), à partir d’assemblages de métadonnées. À l’intérieur des
méta-données nous trouvons des recettes et des bouts de recettes dépendants d’une
recette mère. Une recette correspond à un arbre de compilation. \medskip

Bitbake se charge alors de parcourir (fetch) les sources pour recomposer la recette à travers le 
fichier .bb et les fichiers .bbappend dans un arbre de compilation, puis exécute la compilation.
Cela fait,  il installe tous les binaires dans une même image. L’un des principaux inconvénients de
Yocto, est le besoin de disposer d’un grand espace disque (environ 50 Gb). La première fois, Yocto
a conservé  plusieurs états des tâches. Ainsi, tout ce qui n’a pas été modifié ne sera pas exécuté
à nouveau par Bitbake, on gagne alors un temps précieux en échange de l’espace mémoire
(shared-state cache).

\subsubsection{Poky}

Poky c'est la distribution de référence générée par Yocto, elle est maintenue pour être
compilable sur toutes les machines cibles officielles et se compose :

\begin{itemize}
    \item[-] d'un bootloader (U-boot)
    \item[-] d’un Kernel Linux, et d’applications (GNU compilant pour la plupart)
    \item[-] d’un device tree (fichier binaire .dtb) qui peut être interprété dans /sys/ par des
    drivers
    \item[-] d’un système de fichiers partant du répertoire racine dit rootfs
    \item[-] d’éventuels modules et drivers
\end{itemize}

Poky contient concrètement les lignes de codes nécessaires à l'obtention d'une distribution.

\subsubsection{Bitbake}

Bitbake est un outil de compilation de sources à partir de répertoires locaux et en ligne. Il
décompose chaque phase de compilation en : do\_fetch, do\_unpack, do\_patch,
do\_configure, do\_compile, do\_install, do\_package. Une caractéristique essentielle de
Bitbake est sa capacité à gérer les interruptions dans la compilation et de pouvoir
reprendre la compilation là où il l'avait laissé, à un tâche prête.
Les étapes do\_fetch et do\_unpack respectivement, téléchargent et décompressent les
sources vers le répertoire de travail (variable \$WORKDIR). Pour cela ils passent par un
répertoire intermédiaire de téléchargement (\$DL\_DIR), ces sources peuvent être
téléchargées à la demande par une commande, telle que celle ci-dessous, pour le packet
zlib. En effet, pour des problèmes de connexion réseau, il peut être nécessaire de
déclencher manuellement le téléchargement des paquets non récupérés.

\begin{tcolorbox}
    user@poky:~/fsl-community-bsp/build\$ Bitbake -c fetch zlib
\end{tcolorbox}

Les étapes do\_configure et do\_compile correspondent à la préparation de l'arborescence
de compilation et à la compilation même (précompilation, ln, as...)
des recettes qui composeront l’image. Bitbake sous-traite la préparation et la
compilation sur des commandes comme autotools, cmake, scon, qmake ou encore un
script « ./configure », make, make install; ce choix étant laissé aux rédacteurs du paquet.

Alors que la commande make se contente de placer le fichier Makefile pour ordonner les
compilations, Bitbake apporte plus de dynamisme. Les ordres d’une compilation Bitbake
sont décentralisés sur 3 formats de fichiers. Principalement les .bb sont les fichiers «
recette-mère » qui seront parcourus par Bitbake et déclencheront la compilation (ou non)
des sources en présence. Les fichiers .bbappend (recette-fille) permettent de compléter un
fichier .bb, et les fichiers .bbclass permettent d’indiquer a Bitbake de prendre en compteles
.bb et .bbappend. L'intérêt étant de classer les recettes mères et filles dans des
dossiers (métadonnées ou meta) par fonctionnalité et non par dépendance de compilation.
En constant développement, le “projet Yocto” se compose de poky, Bitbake, ainsi que
l’ensemble des métadonnées mises à disposition par la communauté.

\section{Caméra Raspberry Pi v2}

\begin{figure}[!htb]
    \centering
    \includegraphics[trim={0cm 0cm 0cm 0cm},clip,scale=0.4]{Figures/camrpi.png}
    \decoRule
    \caption{Raspi Cam v2} \label{fig:camrpi}
\end{figure}

La Raspberry Pi Caméra (V2) est la dernière version de la gamme Raspberry. Cette
caméra dispose d’un microcontrôleur imx219 développé par Sony. C’est un composant
d’acquisition d’images associé à un système optique et à quelques composants passifs.
Pour donner une notion de ses capacités, ce microcontrôleur est capable de réaliser une
capture vidéo 1080p à 60 images/s. \medskip

Dans le document suivant nous parlerons toujours du driver imx219 pour désigner le driver
de la caméra Raspberry Pi. La communication avec l’imx219 se fait à travers l’utilisation
de l’I2C sur port CSI2 disposant d’une connexion D-phy 2 ou 4 Lanes et d’une clock pour
le transfert du flux vidéo. Dans notre cas, nous travaillons avec la configuration 2 lanes. La
connexion utilise donc 6 broches avec un doublet (une lane) de pistes en communication
half-duplex et les 2 autres doublets en simplex.\medskip

L’utilisation de l’interface I2C permet de configurer les registres de l’imx219 et de
récupérer le flux vidéo. la communication se fait à une fréquence comprise entre 11 MHz
et 27 MHz.\medskip

Enfin la présence de la broche XCLR permet le reset du composant.
De la documentation technique est accessible pour l’imx219, cependant nous n’avons pas
trouvé la schématique de la carte caméra Raspberry .

\section{Interface caméra MIPI CSI-2}

L’alliance MIPI compte plus de 250 entreprises aujourd’hui mais parmi les 6 contributeurs
initiaux (02/2004) on compte ARM limited, NXP Semiconductors et OmniVision
Technologies AG. Ce dernier est également producteur de contrôleur-caméra. Nous
reparlerons plus tard du composant ov5640 car celui-ci est déjà présent sur la plateforme
openrex au kernel 4.1. \medskip

L’interface processeur des industriels du mobile
standardise les communications avec tous lespériphériques habituels environnant le processeur d’un smartphone. Pour accéder à une
caméra l’interface prévoit une communication par le protocole CSI ou CSI-2 et une liaison
par les couches physiques C-phy et D-phy (M-phy étant en développement). Ce genre de
connexion s’effectue avec 2 à 4 lanes de données et une lane d’horloge. Une lane est un
même signal différentiel à haute fréquence. Dans le standard D-phy, une lane
d’information est transportée sur deux pistes électroniques (en différentiel). Dans le
standard C-phy, 2 lanes d’information peuvent transiter sur 3 pistes électroniques
(doublement différentielles), mais nous ne nous intéresserons pas au C-phy. \medskip

Cette illustration présente les différentes “lane” du block MIPI :

\begin{figure}[!htb]
    \centering
    \includegraphics[trim={0cm 0cm 0cm 0cm},clip,scale=0.4]{Figures/blockMIPI.png}
    \decoRule
    \caption{Différents signaux du MIPI} \label{fig:blockmipi}
\end{figure}

L’imx6s Openrex supporte le mode M-phy 2 lanes. Les deux broches de données et la
broche de la clock seront respectivement les broches (DMO1P/N, DMO2P/N, DCKP/N) du
port CSI-2.

\section{Carte de développement OpenRex}

\begin{figure}[!htb]
    \centering
    \includegraphics[trim={0cm 0cm 0cm 0cm},clip,scale=0.55]{Figures/openrex.png}
    \decoRule
    \caption{Carte OpenRex Basic} \label{fig:openrex}
\end{figure}

L’Openrex est une carte de développement conçue par Fedevel et produite par Voipac.
Elle intègre un « système on chip » IMX6 disposant de un ou quatre coeurs suivant les
versions. La schématique ainsi que le routage de ces deux cartes sont opensources et
disponible sur le site de \href{http://www.imx6rex.com/open-rex/}{http://www.imx6rex.com/open-rex/}.


Notre objectif est de porter le driver imx219 de la Raspberry sur l’IMX6S Openrex. Cette
dernière possède un port CSI-2 identique au port de la Raspberry Pis. De plus, ce port peut accueillir des niveaux de tensions LVDS, nous avons donc vérifié qu’elle était
configurée sur le port CSI-2 comme indiqué par la figure X. Le routage du connecteur
correspond donc à la figure XX? On y retrouve les couples de pistes lane D0, lane D1 et la 
CLK0, les deux broches de communication I2C SCL et SDA, les deux GPIO.

\section{Video for Linux 2 (V4L2)}

V4L2 est la seconde version de l’API V4L. Elle est utilisée par des périphériques vidéo
(caméras, écrans) contenue dans l’espace utilisateur d’un système linux. Elle est aussi
faite de composants audio contrôlés par l’API Alsa. Elle permet de manipuler une très
grande variété de périphériques en utilisant les mêmes fonctions. En général, il s’agit de
périphériques I2C, mais si ce n’est pas le cas, une structure (v4l2\_subdev) a été créée
pour fournir au driver une interface en adéquation avec celle des sous-périphériques.

Principe d'usage :

\begin{itemize}
    \item[-] Ouvrir le périphérique
    \item[-] Modifier les propriétés de l'appareil (résolution, luminosité, ...)
    \item[-] Sélectionner le format de donnée
    \item[-] Récevoir / Envoyer les données
    \item[-] Ferme le périphérique
\end{itemize}

Les drivers V4L2 sont implémentés comme des modules kernels, ils sont chargés
automatiquement ou manuellement (suivant les drivers) lors de la détection du
périphérique par le système. Ils s’exécutent dans le kernel-space c’est-à-dire sous le
kernel.

La commande ci-dessous est une surcouche de la commande insmod. insmod charge
simplement un module, tandis que modprobe charge ses modules dépendants également.

\begin{tcolorbox}
    root@poky:~\# modprobe <driver\_name>
\end{tcolorbox}

Suite au chargement du module, le gestionnaire de périphériques “udev” va créer un
fichier dans le répertoire /dev, permettant ensuite l’accès aux périphériques. Selon le choix
au développement, une arborescence de contrôle plus poussée peut apparaître dans /sys.

\subsection{Package V4L-utils}

Aujourd'hui le package v4l-utils implémente V4L2 dans une distribution , V4L devenant obsolète.

\subsection{Principales commandes de contrôle}

\begin{itemize}
    \item[v4l2-ctl : ]  outil de contrôle utilisé en ligne de commande
    \item[v4l2-dbg : ] outil permettant l'accés aux registres des périphériques v4l2
    \item[q4l2 : ] interface graphique v4l2 utilisant les commandes de contrôle de l'API
\end{itemize}

\begin{figure}[!htb]
    \centering
    \includegraphics[trim={0cm 0cm 0cm 0cm},clip,scale=0.35]{Figures/v4l2.png}
    \decoRule
    \caption{Application graphique de V4L2} \label{fig:v4l2}
\end{figure}

L’API V4L2 met à disposition le package “v4l-utils”. Ce package fournit une interface
graphique et des commandes de contrôle permettant la configuration d’un périphérique. \medskip

Comme il est illustré sur la figure \ref{fig:v4l2}, cette interface permet la configuration
de  plusieurs paramètres disponibles sur une majorité des périphériques vidéo. \medskip

Paramètres généraux : sélection du périphérique d'entrée, configuration du format des données et
dimensionnement de l'affichage.

\begin{itemize}
    \item[Utilisateur : ] Configuration du rendu vidéo
    \item[Caméra : ] Configuration de l'exposition lumineuse
\end{itemize}

Pour manipuler l'ensemble de ces paramètres, les différentes structures de v4l2 doivent
être correctement instanciées dans le driver par les fonctions C, configurant la caméra.

\subsubsection{Structures V4L2}

\textbf{v4l2\_subdev}

\begin{lstlisting}
    struct v4l2_subdev
    {
        #if defined(CONFIG_MEDIA_CONTROLLER)
        struct media_entity entity;
        #endif
        struct list_head list;
        struct module * owner;
        bool owner_v4l2_dev;
        u32 flags;
        struct v4l2_device * v4l2_dev;
        const struct v4l2_subdev_ops * ops;
        const struct v4l2_subdev_internal_ops * internal_ops;
        struct v4l2_ctrl_handler * ctrl_handler;
        char name[V4L2_SUBDEV_NAME_SIZE];
        u32 grp_id;
        void * dev_priv;
        void * host_priv;
        struct video_device * devnode;
        struct device * dev;
        struct device_node * of_node;
        struct list_head async_list;
        struct v4l2_async_subdev * asd;
        struct v4l2_async_notifier * notifier;
        struct v4l2_subdev_platform_data * pdata;
    };
\end{lstlisting}

Cette structure permet de gérer le multiplexage audio et vidéo de
sous-périphériques (capteurs et contrôleurs de caméra).

\textbf{i2c\_client}

\begin{lstlisting}
    struct i2c_client
    {
        unsigned short flags;
        unsigned short addr;
        char name[I2C_NAME_SIZE];
        struct i2c_adapter * adapter;
        struct device dev;
        int irq;
        struct list_head detected;
        #if IS_ENABLED(CONFIG_I2C_SLAVE)
        i2c_slave_cb_t slave_cb;
        #endif
    };
\end{lstlisting}

Cette structure donne accès au bus i2c pour établir la communication et interagir avec le
périphérique. Pour protéger en écriture cette configuration, il est conseillé d’utiliser la
fonction v4l2\_set\_subdevdata(). Elle permet de stocker le pointeur de cette structure dans
les données privées de v4l2\_subdev.

\textbf{Initialisation de v4l2\_subdev}

Déclaration des fonctions d’initialisation dans la structure v4l2\_subdev\_core\_ops

\begin{lstlisting}
    static struct v4l2_subdev_core_ops imx219_subdev_core_ops =
    {
    .s_power = imx219_s_power,
    };
\end{lstlisting}

Implémentation des fonctions permettant l’initialisation des paramètres du flux vidéo :

\begin{lstlisting}
    static struct v4l2_subdev_video_ops imx219_subdev_video_ops = 
    {
    .s_stream = imx219_s_stream,
    .cropcap = imx219_cropcap,
    .g_crop = imx219_g_crop,
    .s_crop = imx219_s_crop,
    .enum_mbus_fmt = imx219_enum_mbus_fmt,
    .g_mbus_fmt = imx219_g_mbus_fmt,
    .try_mbus_fmt = imx219_try_mbus_fmt,
    .s_mbus_fmt = imx219_s_mbus_fmt,
    .g_mbus_config = imx219_g_mbus_config,
    };
\end{lstlisting}

On crée donc la structure du périphérique (imx219.c) :

\begin{lstlisting}
    struct <chipname>_state 
    {
        struct v4l2_subdev sd;
    };
\end{lstlisting}

Cette structure doit contenir la structure v4l2\_subdev pour donner un accés direct à la
configuration du sous-périphérique (interface i2c).

Initialiser le sous-périphérique i2c :

\begin{lstlisting}
    v4l2_i2c_subdev_init(&state->sd, client, subdev_ops);
\end{lstlisting}

Faire le lien entre la structure i2c et v4l2\_subdev :

\begin{lstlisting}
    struct i2c_client *client = v4l2_get_subdevdata(sd);
    struct v4l2_subdev *sd = i2c_get_clientdata(client);
\end{lstlisting}

Instancier la structure pour ajouter la configuration du périphérique au kernel :

\begin{lstlisting}
    struct v4l2_subdev * v4l2_i2c_new_subdev(struct v4l2_device * v4l2_dev, struct
    i2c_adapter * adapter, const char * client_type, u8 addr, const unsigned short *
    probe_addrs);
\end{lstlisting}

Charge la configuration du flux vidéo du port CSI-2 définit par l’utilisateur.

\begin{lstlisting}
    v4l2_subdev_video_ops->s_stream()
\end{lstlisting}

Mise sous tension du port CSI-2 :

\begin{lstlisting}
    v4l2_subdev_core_ops->s_power()
\end{lstlisting}

\section{GStreamer}

GStreamer est un framework multimédia développé en C et porté sur plusieurs autres
systèmes d’exploitation que GNU/Linux comme Android, OS X, iOS ou encore Windows.
Ce projet débuta en Juin 1999 et fut implémenté dans l’environnement bureautique
GNOME en Juillet 2012. \medskip

GStreamer utilise principalement des tubes (pipeline) inter-connectés ainsi le type d’un
flux traversant un tube est connu des autres de plus ce framework est capable de gérer
des fichiers audio et vidéo (capture, encodage, streaming, écoute, affichage).\medskip

GStreamer est basé sur des plugins qui améliorent son développement et ajoute des
fonctionnalités comme l’encodage/décodage géré par le plugin FFMPEG.\medskip

GStreamer propose une fonctionnalité de streaming en local ou par un réseau, cette
dernière passe par les protocoles UDP et TCP de la couche IP.\medskip

Le principe de ce framework repose sur l’association d’éléments reliés par des pipelines
cependant l’entrée d’un élément doit être compatible avec la sortie de l’élément précédent.
L’ordre des éléments est donc très important lorsqu’on écrit la commande, voici un
schéma expliquant l’ordre des éléments :

\begin{figure}[!htb]
    \centering
    \includegraphics[trim={0cm 5cm 0cm 6cm},clip,scale=0.35]{Figures/gstreamer.png}
    \decoRule
    \caption{Étapes de GStreamer} \label{fig:gstreamer}
\end{figure}

La première étape indique la source de la commande GStreamer. Cela peut être un fichier
ou bien une caméra par le biais de V4L2. Ensuite, le système détecte le type du fichier
source et différencie un fichier audio d’un fichier vidéo.
Nous nous intéresserons plus sur la partie vidéo qu’audio puisque l’objectif du projet est la
capture du flux vidéo, pour cela nous allons expliquer les commandes que nous utilisions :

\begin{tcolorbox}
    user@poky:~\$ gst-launch-1.0 v4l2src ! 'image/jpeg,width=1280,height=720,
    framerate=30/1' ! imxvpudec ! imxipuvideotransform ! imxeglvivsink sync=false
\end{tcolorbox}

Si l’on souhaite visualiser le flux vidéo d’un appareil V4L2 avec la synchronisation
désactivé en 30 FPS et d’une résolution de 1280x720 pixels avec un format d’image
JPEG, cette commande est adaptée. De plus celle-ci utilise une synchronisation EGL ainsi
qu’un décodage V4L2.

\begin{tcolorbox}
user@poky:~\$ gst-launch-1.0 v4l2src ! 'image/jpeg,width=1280,height=720,
framerate=30/1' ! v4l2sink sync=false
\end{tcolorbox}

Cette commande permet d’afficher le flux vidéo de V4L2 en 30 FPS avec une taille de
1280x720 pixels. Si l’on souhaite juste faire une capture photo, l’image sera au format
JPEG. V4L2sink est utilisé pour afficher le flux vidéo de l’appareil V4L2 en désactivant la
synchronisation.

\begin{tcolorbox}
user@poky:~\$ gst-launch-1.0 v4l2src device=/dev/video0 ! 'image/jpeg,
width=1280,height=720, framerate=30/1' ! v4l2sink sync=false
\end{tcolorbox}

Cette commande est la même que la première, la seule différence vient de la source.
Celle-ci prend l’appareil vidéo numéro 0 en entrée qui correspond à la caméra. 
% Chapter 3

\chapter{Travail réalisé} % Main chapter title

\label{Chapter3} % For referencing the chapter elsewhere, use \ref{Chapter3} 

%----------------------------------------------------------------------------------------

\section{Génération d'une méta-donnée}

Pour réaliser une métadonnée nous avons lancé la commande suivante :

\begin{tcolorbox}
    user@poky:~\$ yocto-layer create-layer your\_layer\_name
\end{tcolorbox}

Elle nous assiste lors de la génération d’une grande partie de l’architecture standardisée
commune aux metas. \medskip 

Nous avons ensuite comparé notre meta à la meta-skeleton, le squelette de base
permettant de doter notre système d’exploitation d’un programme helloworld. \medskip 

Nous aurions préféré partir de la commande ci-dessous seulement celle-ci n’est
accessible avant la version 2.4 du projet.

\begin{tcolorbox}
    user@poky:~\$ bitbake-layer create-layer your\_layer\_name
\end{tcolorbox}

Nous rajoutons cette meta à la liste des objets à compiler par la commande suivante.

\begin{tcolorbox}
    user@poky:~\$ bitbake-layer add-layer your\_layer\_name
\end{tcolorbox}

Après utilisation voici à quoi la meta ressemble depuis l’interface de développement atom.

\begin{figure}[!htb]
    \centering
    \includegraphics[trim={0cm 0cm 0cm 0cm},clip,scale=0.31]{Figures/architecture.png}
    \decoRule
    \caption{Architecture d'une méta-donnée} \label{fig:architecture}
\end{figure} 

Au début de la création de notre métadonnée nous souhaitions pouvoir reconstituer notre
espace de travail aisément et cela est possible grâce aux deux dépôts
fsl-community-bsp-platform et fsl-community-bsp-base de Freescale. Cependant il est
nécessaire de les modifier afin qu’ils correspondent à aux fichiers nécessaires à notre
meta-donnée. \medskip 

Le dépôt fsl-community-bsp-platform contient un fichier “default.xml” que l’on appelle un
fichier manifest, celui-ci permet de définir tous les dépôts contenant les fichiers et dossiers
nécessaires à la création de notre image Yocto. \medskip 

Le second dépôt fsl-community-bsp-data créera notre fichier “bblayers.conf” \\
avec toutes les métadonnées nécessaires à la compilation ainsi que le fichier \\
“setup-environment” qui concevra notre fichier local.conf.

\subsection{fsl-community-bsp-platform}

Nous avons repris le “default.xml” utilisé par Frescale incluant un kernel 4.1
\href{http://git.freescale.com/git/cgit.cgi/imx/fsl-arm-yocto-bsp.git/tree/default.xml?h=imx-4.1-krogoth}
{(sur ce lien)}

Cela a permis de connaître les dépôts ainsi que leurs branches afin de réaliser notre
propre fichier manifest. Nous avons rajouté les liens vers notre méta-donnée ainsi que
notre dépôt fsl-community-bsp-data grâce à ces deux lignes : \medskip

\begin{lstlisting}
    <remote fetch="git://github.com/petit-romain" name="romain"/>
    <remote fetch="git://github.com/Alanaitali" name="equipe"/>    
\end{lstlisting}

On a donné l’ordre de récupérer le fichier “setup-environment” permettant de sourcer notre
environnement de travail en le plaçant dans le dossier sources/base : \medskip 

\begin{lstlisting}
    <project remote="romain" revision="master" name="fsl-community-bsp-base" path="sources/base">
        <copyfile dest="setup-environment" src="setup-environment"/>
    </project>
\end{lstlisting}

Enfin on précise la branche sur laquelle nous développons notre meta ainsi que son
emplacement dans notre dossier de travail (sources/meta-openrexpicam) :

\begin{lstlisting}
    <project remote="equipe" revision="develop/porting" path="sources/meta-openrexpicam"/>
\end{lstlisting}

\subsection{fsl-community-bsp-data}

Ce dépôt recopiera automatiquement notre fichier “bblayers.conf” puisqu’il \\ 
contient le contenu du fichier : 

\begin{lstlisting}
    BBLAYERS = " \
    ${BSPDIR}/sources/poky/meta \
    ${BSPDIR}/sources/poky/meta-Yocto \
    \
    ${BSPDIR}/sources/meta-openembedded/meta-oe \
    ${BSPDIR}/sources/meta-openembedded/meta-multimedia \
    \
    ${BSPDIR}/sources/meta-fsl-arm \
    ${BSPDIR}/sources/meta-fsl-arm-extra \
    ${BSPDIR}/sources/meta-fsl-demos \
    \
    ${BSPDIR}/sources/meta-openrexpicam \
    "
    BBLAYERS += "${BSPDIR}/sources/meta-fsl-arm-voipac"
\end{lstlisting}

Pour le fichier “setup-environment”, nous n’avions rien à modifier cependant nous avons
ajouté une signature ainsi que les images compilables de notre méta-donnée :

\begin{lstlisting}
    Welcome to our project !
    You can now run 'bitbake <target>'
    Common targets are :
        - openrexpicam-base-image
    Signed-off-by: Alan Ait-Ali, Romain Petit, Clément Ailloud & Martin Laporte
\end{lstlisting}

\section{Génération d'un OS}

\subsection{Préparation de l'environnement}

Grâce à l’environnement de travail Yocto nous avons pu rapidement mettre en place un
système d’exploitation fonctionnel sur la cible Openrex. En effet les équipes de Voipac et
Fedevel ont mis à disposition le support de la carte (BSP) afin que la distribution
GNU/Linux maintenue par Freescale pour les processeurs imx6 fonctionne sur les
Openrex. L’ensemble de sources fsl-community-bsp supporte donc la carte
Openrex-imx6q. Cette officialité permet, une fois les sources correctement téléchargées et
ordonnées, de compiler une image par la commande :

\begin{tcolorbox}
    MACHINE=imx6s-openrex bitbake core-image-base
\end{tcolorbox}

\subsection{Préparation de notre meta-donnée}

Le projet Yocto permet aussi d’améliorer l’image générée. Cette amélioration suit le principe des
codes ouverts à l’amélioration et fermés aux modifications. En effet comme représenté sur la
figure \ref{fig:recette}, lors de la compilation des recettes (100 Mo) nécessaires à la core-image-base, le
programme bitbake a puisé dans des sources distantes pour composer un dossier de sources à compiler. 
Grâce à celles-ci bitbake forme un répertoire général des sources du noyau (18Go), bitbake compile 
ensuite la core-image-base et la place dans :  \medskip

\$BUILDDIR/tmp/deploy/images/{NOM\_DE\_L’IMAGE:imx6-openrexbasic} \medskip

Mais pour modifier cette image il n’y a pas besoin de toucher aux recettes des
meta-données initiales (100Mo sur la figure \ref{fig:recette}). Pour modifier l’image on intervient
d'abord au niveau du dossier des sources (18Go) on apporte nos améliorations. De ces
modifications on réalise un patch différenciant l’état d’origine et l’état actuel des fichiers
voulus. Enfin on vient placer ce patch dans notre propre meta (1Mo) à côté des
meta-données originales. En visionnant cette meta et les patchs qui s’y trouvent on
versionne donc l’ensemble du projet depuis l’état d’origine fixé par la communauté
freescale.

\begin{figure}[!htb]
    \centering
    \includegraphics[trim={0cm 0cm 0cm 0cm},clip,scale=0.1]{Figures/recette.png}
    \decoRule
    \caption{Architecture d'une recette} \label{fig:recette}
\end{figure} 

\section{Implémentation de code dans le BSP}

Le coeur du projet vise à ajouter du code dans le bsp déjà existant, notamment au sein du
kernel. Pour ce faire nous modifions le dossier de compilation (18Gb ci-dessus). Comme
son nom l’indique, /git/ est versionné par git et nous permet de tirer un patch de nos
modifications. \medskip

Ensuite, on applique des patchs modifiant les sources du kernel provenant
du git de voipac ou fedevel. le patch généré est inclu aux sources de notre recette
linux-voipac\_\%.bbappend. La recette s'ajoutera à la fin de la recette linux-voipac lors de
sa prise en compte par bitbake. \medskip

La recette linux-voipac\_\%.bbappend a pour rôle de surcharger la recette se trouvant dans
la meta-fsl-arm-voipac. Elle porte le même nom que la recette a surcharger suivi d’un \_\%
qui permet de s’affranchir de la version. \medskip

Dans l’extrait ci-dessous on peut voir que la recette utilise les sources du git voipac.

\subsection{Extrait de linux-voipac-4.1.bb}

\begin{lstlisting}
SRCBRANCH = "4.1-2.0.x-imx-rex"
LOCALVERSION = "-Yocto"
SRCREV = "ab5923c9613a97ede4da92a933842e771283d463"
KERNEL_SRC ?= "git://github.com/voipac/linux-fslc.git;protocol=git"
SRC_URI = "${KERNEL_SRC};branch=${SRCBRANCH} file://defconfig"
\end{lstlisting}

Notre recette permet d’ajouter les fichiers se trouvant dans imx6-openrexbasic et inscrit
dans la recette comme on le voit ci-dessous.

\subsection{Extrait de linux-voipac\_ \%.bbappend}

\begin{lstlisting}
    FILESEXTRAPATHS_prepend := "${THISDIR}/${PN}:"
    SRC_URI_append_imx6-openrexbasic += " \
        file://0001-imx219.patch \
        file://defconfig \
    "
\end{lstlisting}

Quand nous modifions le code du dossier de compilation, le patch se réfère à l’état dans
lequel la communauté des bsp freescale l’a laissé. Or dans le fichier
\$BUILDDIR/conf/bblayer.conf notre meta-openrexpicam est placée directement en suivant
des meta nécessaires pour atteindre l’état laissé par freescale. Nous sommes donc sûrs
que le patch généré est cohérent avec l’arborescence de compilation sur laquelle bitbake
l’appliquera.

\section{Implémentation des supports de compilation}

Ajouter une nouvelle caméra a notre BSP, demande la création d’un driver. Pour qu’il soit
utilisable il nous faut signaler au compilateur que nous voulons ajouter à notre kernel le
support pour la caméra. Pour paramétrer la compilation on passe par l’outil menuconfig
qui permet 3 options :

\begin{itemize}
    \item[-] compilation du driver et chargement en module
    \item[-] compilation du driver et chargement en statique
    \item[-] pas de compilation du driver
\end{itemize}

Les fichiers qui permettent de paramétrer le menuconfig sont situés au même endroit que
les fichiers source.c des pilotes sous le nom de Kconfig.

L’ajout de notre caméra dans menuconfig se fait par le code suivant :

\begin{lstlisting}
    config MXC_CAMERA_IMX219_MIPI
    tristate "Sony imx219 camera support using mipi (raspicam v2)"
    depends on !VIDEO_MXC_EMMA_CAMERA && I2C
\end{lstlisting}

\begin{itemize}
    \item[depends on : ] Définit les modules à activer pour que l’option soit visible
    \item[tristate : ] Définit l’affichage dans le menuconfi
    \item[Config : ] Définit le nom de la variable de compilation
\end{itemize}

\begin{figure}[!htb]
    \centering
    \includegraphics[trim={0cm 0cm 0cm 0cm},clip,scale=0.1]{Figures/menuconfig.png}
    \decoRule
    \caption{Driver dans le menuconfig} \label{fig:menucfg}
\end{figure}

Sur la figure ci-dessus on peut remarquer un M juste devant les drivers, cela signifie qu’ils
sont compilés en module. Nous avons choisi cette option afin de pouvoir décharger
(rmmod) et recharger (modprobe) le driver sans devoir redémarrer le kernel.
Le résultat du menuconfig est un fichier texte .config
(\$BUILDDIR/tmp/work/imx6\_openrexbasic-poky-linux-gnueabi/linux-voipac/4.1-r0/build/ex
emple.config) ou généralement defconfig. Dans l’extrait du defconfig ci-dessous on peut
observer que l’ajout et le chargement en module du pilote a été pris en compte.

\begin{lstlisting}
    CONFIG_VIDEO_MXC_IPU_CAMERA=y
    CONFIG_MXC_CAMERA_OV5640=m
    CONFIG_MXC_CAMERA_OV5642=m
    CONFIG_MXC_CAMERA_OV5640_MIPI=m
    CONFIG_MXC_CAMERA_OV5647_MIPI_INT=m
    CONFIG_MXC_TVIN_ADV7180=m
    CONFIG_MXC_TVIN_ADV7610=m
    CONFIG_MXC_CAMERA_IMX219_MIPI=m
    CONFIG_MXC_IPU_DEVICE_QUEUE_SDC=m
    CONFIG_MXC_IPU_PRP_ENC=m
    CONFIG_MXC_IPU_CSI_ENC=m
\end{lstlisting}

\begin{itemize}
    \item[CONFIG\_MXC\_CAMERA\_IMX219\_MIPI : ] Objet du kernel
    \item[m] Ordre de compilation et d'installation en module
\end{itemize}

Le menuconfig permet de configurer la compilation, mais il n’est pas directement lié à
celle-ci, c’est le Makefile qui se charge de relier les informations contenues dans le .config
au compilateur. Comme le Kconfig, le Makefile se trouve dans le répertoire des sources :
\$BUILDDIR/tmp/work/{imx6\_openrexbasic-poky-linux-gnueabi/{recette}/4.1-r0/git/drivers/media/{Selon le driver}

Selon le driver, ses sources peuvent prolonger ce chemin vers “platform/mxc/capture/” ou
“/i2c/.”. Le nom de la recette, lui dépends du kernel compilé. La recette “linux-openrex” est
utilisée pour les kernels v3 et “linux-voipac” pour les kernels v4.

\begin{lstlisting}
    +imx219_camera_mipi-objs := imx219_mipi.o
    +obj-$(CONFIG_MXC_CAMERA_IMX219_MIPI) += imx219_camera_mipi.o
\end{lstlisting}

\begin{itemize}
    \item[-] imx219\_camera\_mi-objs : nom du fichier .c à compiler
    \item[-] CONFIG\_MXC\_CAMERA\_IMX219\_MIPI : variable de compilation fourni par le .config
    \item[-] imx219\_camera\_mipi.o nom du driver compilé
\end{itemize}

Au lancement de bitbake le driver pourra alors être compilé et chargé par le kernel.

\section{Utilsation de drivers existants}

\subsection{Présentation d'un driver}

Un driver Linux est un programme binaire qui s’exécute dans le kernel-space. Un driver
utilise les ABI kernel pour interagir avec son environnement. Le code d’un driver s’appuie
donc sur les API kernel. Dans le cas du système d’exploitation linux, celles-ci sont
rédigées en langage c. C’est pourquoi il est nécessaire que le compilateur du driver
compile parfaitement le langage c. Dans notre cas comme dans la majorité, le driver sera
écrit en c. Une particularité des codes de driver provient de l’absence de main(), celui-ci
est remplacé par des fonctions init() et exit(). Init() permet le chargement du driver dans le
kernel. Si le driver est compilé en statique, il est chargé (linked and locked) au démarrage
et exit() est exécuté lors de l’extinction du système. Si le driver est compilé en module ce
sont les fonctions insmod et rmmod qui appelleront les init et exit du fichier “.ko”

\subsection{Utilisation de l'existant}

Dans un premier temps, n’étant pas habitués à manipuler des drivers, nous \\ avons
cherché à utiliser des sources en croisant les architectures requises. Pour agir dans notre meta,
nous avons eu recours à une série de patchs comme montrés ci-dessous.

\begin{figure}[!htb]
    \centering
    \includegraphics[trim={0cm 0cm 0cm 0cm},clip,scale=0.35]{Figures/patchs.png}
    \decoRule
    \caption{Patchs du kernel 3.14} \label{fig:patchs}
\end{figure} 

\subsubsection{Compilation par SDK}

Dans un premier temps, nous nous doutions qu’il y aurait des erreurs de compilation.
Nous avons préféré décorréler les erreurs de compilation de nos propres erreurs. Nous
avons alors essayé de compiler ces fichiers, en dehors de la meta depuis la
cross-toolchain générée par Yocto. Nous utilisons la variable \$CC sélectionnant le cross
compilateur, cette variable est affectée lorsque l’on “source” le sdk généré par :

\begin{tcolorbox}
    \#génération du sdk
    bitbake openrexpicam-base-image -c populate\_sdk
    \#environement du sdk
    source /opt/poky/2.0.3/environment-setup-cortexa9hf-vfp-neon-poky-linux-gnueabi
    \#aperçu de CC
    CC=arm-poky-linux-gnueabi-gcc -march=armv7-a -marm -mthumb-interwork -mfloat-abi=hard -mfpu=neon
    -mtune=cortex-a9 –sysroot=/opt/poky/2.0.3/sysroots/cortexa9hf-vfp-neon-poky-linux-gnueabi
    \#première compilation « out-of-tree »
    \$CC imx219.c
\end{tcolorbox}

Cette compilation fait appel à des bibliothèques contenues dans les arborescences de
compilation de différentes recettes. 

\begin{tcolorbox}
    BUILDDIR=/home/diag/workspaceThales/YOCTO/fsl-community-bsp/build-imx6rex.com
    IMX6S\_DIR=\$BUILDDIR/tmp/work/imx6s\_openrex-poky-linux-gnueabi
    \#libs linux/*.h
    DIRLINUX=\$IMX6S\_DIR/linux-openrex/3.14-r0/git/include
    DIRLINUX2=\$IMX6S\_DIR/u-boot-openrex/v2015.10+gitAUTOINC+7d8ddd7de7-r0/git/include
    DIRLINUX3=\$IMX6S\_DIR/core-image-minimal/1.0-r0/sdk/image/opt/poky/2.0.3/sysroots/cortexa9hf-vfp-neon
    -poky-linux-gnueabi/usr/include
    \#libs asm/*.h
    DIRASM=\$BUILDDIR/tmp/work/x86\_64-nativesdk-pokysdk-linux/nativesdk-linux-libc-headers/4.1-r0/linux-4.1
    /arch/arm/include/
    DIRASM2=\$IMX6S\_DIR/linux-openrex/3.14-r0/build/arch/arm/include/generated
    DIRASM3=\$IMX6S\_DIR/u-boot-openrex/v2015.10+gitAUTOINC+7d8ddd7de7-r0/git/arch/arm/include/
    \#nouvelle commande de compilation « out of tree »
    \$CC imx219.c -I\$DIRLINUX -I\$DIRASM -I\$DIRASM2 -I\$DIRLINUX2 -I\$DIRLINUX3 -I\$DIRASM3
\end{tcolorbox}

Nous avons inclus 6 dossiers de bibliothèques en option puis les nouvelles dépendances
étaient inexistantes dans le dossier contenant l’arborescence de compilation. Donc nous
nous sommes mis à chercher une autre solution. Après avoir parlé à notre professeur de
Linux embarqué des soucis de compilation que nous rencontrions nous nous sommes mis
à compiler le driver en ajoutant ses sources dans l’arborescence (in-tree).

\subsubsection{Compatilation out-of-tree}

Pour cette première compilation de code source dans Yocto nous avons tout d'abord
rédigé une recette comme on aurait rédigé un makefile. Un intérêt est de pouvoir compiler
nos sources sans avoir à recompiler le kernel entier. Seulement cette méthode ne résout
pas le problème de gestion des dépendances. En effet même si Yocto compile avec la
même configuration le kernel et ce module, Yocto interprètera les recettes comme deux
compilations différentes et les exécutera dans des répertoires dissociés. les includes du
module seront incapables de trouver plus automatiquement que de manière out-of-tree
leurs bibliothèques kernel (linux/example.h).

\begin{figure}[!htb]
    \centering
    \includegraphics[trim={0cm 0cm 0cm 0cm},clip,scale=0.35]{Figures/outtree.png}
    \decoRule
    \caption{Arborescence de la compilation out-of-tree} \label{fig:outtree}
\end{figure} 

\subsubsection{Compatilation in-tree}

La compilation de ces sources a été effectuée in-tree sur la version v3.14 du kernel
linux-openrex (branche develop/Smart). La compilation ayant fonctionné, elle a été testée
et déclarée image non bootable. Comme à ce moment là les sources de type chromium os
étaient plus avancées, cette piste fut abandonnée.

\subsection{imx219 - Hummingboard}

\subsubsection{Origine des sources}

Comme nous n’avions pas encore manipulé un driver par son code source nous sommes
rapidement allés demander des conseils aux professeurs renseignés. Principalement M
P.-J. Texier qui nous gardera à l’oeil le long de ce projet. Dans un mail à notre équipe M
Texier nous a proposé de s’inspirer du driver disponible à l’adresse suivante. Il s’agit du git
“Russell King's ARM Linux kernel tree” mais surtout d’un patch qui ajoute le driver
d’imx219 et configure le makefile et kconfig pour sa compilation. L’intérêt principal étant la
plateforme cible, une “hummingboard” portant un soc imx6dl, différent mais proche de
notre imx6s.
De plus sur ce même kernel tree, on peut trouver les fichiers device-tree correspondants
au driver pour la configuration en question (
\href{http://git.armlinux.org.uk/cgit/linux-arm.git/commit/?h=csi-v6&id=e3f847cd37b007d55b76282414bfcf13abb8fc9a}
{source}, 
\href{http://git.armlinux.org.uk/cgit/linux-arm.git/commit/?h=csi-v6&id=4bd8e1231a2e6eca6a65b565176ea9722611c8dd}
{dts}).

Lors des rapports intermédiaires présentés à l’équipe de Thales nous parlions de ces
sources sous le nom de “solution 3”

\subsubsection{Compilation in-tree}

Sur les conseils de M Texier, nous n’avons pas cherché à réaliser de compilation
out-of-tree. Nous avons donc directement porté le driver sur notre carte ainsi que les
structures device-tree.

\subsection{imx219 - Nvidia-Tegra Chromium-Os}

Chromium OS est “un système d’exploitation visant à renforcer la sécurité des utilisateurs qui
passent une majorité de leur temps sur le web”. Concrètement cet os est un système GNU/Linux et
correspond à la branche en développement libre de “Google Chrome os”. Développé par par une filiale
de Google, cet os est entre autre maintenu par Nvidia sur ses processeurs Tegra. Ceux-ci
sont destinés à des applications dans le domaine des smartphones ; ces processeurs
respectent très probablement à la lettre les contraintes MIPI.

\subsubsection{Origine des sources}

Le driver en question provient du git officiel de chromium os de nvidia pour les cartes
tegra, sur le kernel 3.14. il est accessible à ce
\href{https://chromium.googlesource.com/chromiumos/third_party/kernel/+/factory-ryu-6486.14.B-chromeos-3.14/drivers/media/}{lien} .
Lors des rapports intermédiaires présentés à l’équipe de Thales nous parlions de ces
sources sous le nom de “solution 2”.

\subsubsection{Compatibilité avec le device tree}

Lorsque l’on essayait de charger le driver imx219, nous obtenions l’erreur suivante

\begin{tcolorbox}
    imx219 1-0064: IMX219: missing platform data !
\end{tcolorbox}

Après analyse du driver l’erreur provient de la fonction probe du driver. Cette erreur est
survenue dû à une mauvaise compatibilité entre le driver imx219 et le gestionnaire I2C,
pour cela nous avons ajouté une compatibilité avec le device-tree de cette façon :

\begin{lstlisting}
    static const struct of_device_id imx219_of_match [] =
    {
        { . compatible = " sony , imx219 " , . data = 0 } ,
        {}
    };
    MODULE_DEVICE_TABLE ( of , imx219_of_match ) ;
\end{lstlisting}

Ensuite nous avons créé notre propre structure I2C pour l’imx219 afin d’implémenter la
compatibilité entre le driver et l’I2C de cette manière :

\begin{lstlisting}
    static struct i2c_driver imx219_i2c_driver =
    {
        . driver =
            {
                . name = " imx219 " ,
                . of_match_table = of_match_ptr(imx219_of_match) ,
            },
        . probe = imx219_probe ,
        . remove = imx219_remove ,
        . id_table = imx219_id ,
    };
\end{lstlisting}

De plus nous avons enlevé la condition qui nous procurait l’erreur précédente afin de
tester dans un premier temps si la compilation s'effectuait correctement :

\begin{lstlisting}
    if (! ssdd)
    {
        dev_err(& client->dev, "IMX219:missing platform data !\ n");
        return -EINVAL ;
    }
\end{lstlisting}

Après avoir modifié le driver nous obtenons une nouvelle erreur :

\begin{tcolorbox}
    imx219 1-0064: Error -19 getting clock \\
    i2c 1-0064: Driver imx219 requests probe deferral
\end{tcolorbox}

\begin{lstlisting}
    if (! ssdd)
    {
        dev_err(& client->dev, "IMX219:missing platform data !\ n");
        return -EINVAL ;
    }
\end{lstlisting}

Nous avons repris la même stratégie que précédemment en enlevant la condition
générant l’erreur :

\begin{lstlisting}
    if(IS_ERR(priv->clk))
    {
        dev_info(&client->dev,"Error %ld getting clock \ n",
        PTR_ERR(priv->clk));
        return -EPROBE_DEFER ;
    }
\end{lstlisting}

Enfin la capture d’écran ci-dessous permet de justifier que le module se charge bien au
lancement de notre OS :

\begin{tcolorbox}
    root@openrexpicam:~\# i2cdetect -y 1 \\
        \hspace{0.8cm}0\hspace{0.3cm}1\hspace{0.3cm}2\hspace{0.3cm}3\hspace{0.3cm}4\hspace{0.3cm}5
        \hspace{0.3cm}6\hspace{0.3cm}7\hspace{0.3cm}8\hspace{0.3cm}9\hspace{0.3cm}a\hspace{0.3cm}b
        \hspace{0.3cm}c\hspace{0.3cm}d\hspace{0.3cm}e\hspace{0.3cm}f \\
    00: - - - - - - - - - - - - - - - - - - - - - - - - - - - - - - - - \\
    10: 10  - - - - - - - - - - - - - - - - - - - - - - - - - - - - - - \\
    20: - - - - - - - - - - - - - - - - - - - - - - - - - - - - - - - - \\
    30: - - - - - - - - - - - - - - - - - - - - - - - - - - - - - - - - \\
    40: 40  - - - - - - - - - - - - - - 48  - - - - - - - - - - - - - - \\
    50: uu  - - - - - - - - - - - - - - - - - - - - - - - - - - - - - - \\
    60: - - - - - - - - - - uu  - - - - - - - - - - - - - - - - - - - - \\
    70: - - - - - - - - - - - - - - - - - - - - - - - - - - - - - - - - \\
    
    root@openrexpicam:~\# lsmod \\
    Module                      \hspace{2.5cm}Size        \hspace{2cm}Used by\\
    mxc\_v4l2\_capture          \hspace{0.8cm}25109       \hspace{1.75cm}0 \\
    ipu\_bg\_overlay\_sdc       \hspace{0.5cm}5242        \hspace{1.9cm}1 mxc\_v4l2\_capture\\
    ipu\_still                  \hspace{2.5cm}2312        \hspace{1.95cm}1 mxc\_v4l2\_capture\\
    ipu\_prp\_end               \hspace{1.65cm}5872       \hspace{1.95cm}1 mxc\_v4l2\_capture\\
    ipu\_csi\_enc               \hspace{1.9cm}3743        \hspace{1.95cm}1 mxc\_v4l2\_capture\\
    v4l2\_int\_device           \hspace{1.23cm}2913       \hspace{1.95cm}2 ipu\_csi\_enc, mxc\_v4l2\_capture\\
    ipu\_fg\_overlay\_sdc       \hspace{0.55cm}6068       \hspace{1.95cm}1 mxc\_v4l2\_capture\\
    imx219                      \hspace{2.7cm}7716        \hspace{1.95cm}1 \\
    mxc\_dcic                   \hspace{2.3cm}6543        \hspace{2cm}0\\
    galcore                     \hspace{2.65cm}225000     \hspace{1.65cm}0\\
    evbug                       \hspace{2.8cm}1871        \hspace{2.05cm}0\\
\end{tcolorbox}

Une fois ceci effectué nous voulions essayer de corriger les instructions conditionnelles de
sécurité générant les erreurs afin de pouvoir charger correctement le driver. Une fois le
driver chargé, nous n’avons pas réussi à effectuer la liaison entre v4l2 et le driver imx219.
Par manque de temps et de connaissances sur l’environnement v4l2, nous nous sommes
tous concentré sur la nouvelle solution proposé par Thalès qui leur semblait plus
pertinente.

\subsection{imx219 - Raspberry Pi v2}

\subsubsection{Origine des sources}
En premier lieu nous sommes partis des sources reçues dans les premiers \\
mails. Ce driver lie le processeur Allwinner A80 (ARM Cortex-A15/A7) du sbc Raspberry Pi à un
composant pilote vidéo imx219. Ces deux fichiers Linux V4L2 Driver (imx219.c et
camera.h) onts été rédigés par “Chomoly“ et sont téléchargeables à cette 
\href{https://www.raspberrypi.org/forums/viewtopic.php?f=43&t=162722}{adrese}. \medskip

Lors des rapports intermédiaires présentés à l’équipe de Thales nous parlions de ces
sources sous le nom de “solution 1”. \medskip

Chronologiquement, ces fichiers sources ont été la première piste explorée c’est pourquoi
leur compilation n’a pas été immédiatement réussie. Nous avons essayé d’obtenir un
module binaire par trois manières. D’abord, une compilation grâce à un sdk yocto, puis
une compilation out-of-tree, externe à la recette du kernel et nous sommes presque
parvenus à nos fins avec une compilation in-tree. Ci-dessous, la description des pratiques
à éviter pour le projet d’un driver.

\subsubsection{Portage du BSP Voipac sur un kernel 4.14}

Lors de nos recherches pour les compilations ci-dessous, nous avons trouvé un driver
imx219 fonctionnel sur un kernel 4.14 donc l’idée était de porter notre méta-donnée vers
un kernel 4.14 afin de réutiliser le driver sur cette version de kernel.
Nous sommes partis de la même structure que les versions de kernel 3.14 et 4.1 qui
contient un dossier avec le defconfig et les différents patchs nécessaires à la compilation
ainsi qu’un fichier .bb.

\begin{tcolorbox}
    user@poky~\$ : tree sources/meta-fsl-arm-voipac/recipes-kernel/linux \\
    |\_\_linux-voipac-3.14 \\
    |\_\_\_\_\_defconfig \\
    |\_\_linux-voipac\_3.14.bb \\
    |\_\_linux-voipac-4.1 \\
    |\_\_\_\_\_defconfig \\
    |\_\_linux-voipac\_4.1.bb \\
    |\_\_linux-voipac-4.14 \\
    |\_\_\_\_\_0001-imx6s-6q-add-initial-support.patch \\
    |\_\_\_\_\_defconfig \\
    |\_\_linux-voipac\_4.14.bb \\
\end{tcolorbox}

Ensuite nous avons modifié la recette du kernel 4.1 dans une nouvelle recette pour un
kernel 4.14 avec les modifications suivantes :

\begin{lstlisting}
    SRCBRANCH = "4.14.x+fslc"
    LOCALVERSION = "-Yocto"
    SRCREV = "${AUTOREV}"
    KERNEL_SRC ?= "git://github.com/Freescale/linux-fslc.git;protocol=git"
\end{lstlisting}

Cependant lors de la phase de boot nous obtenons l’erreur suivante :

\begin{tcolorbox}
    reading boot-imx6-openrexbasic.scr \\
    ** Unable to read file boot-imx6-openrexbasic.scr **
\end{tcolorbox}

Nous pouvons conclure qu’il y a une incompatibilité entre le bootloader et le kernel que
nous avons modifié. Pour que notre image puisse se lancer correctement, il est nécessaire
de modifier le bootloader.

\section{Développement d'un driver}

Des caméras utilisant le MIPI CSI sont implémentées de base dans le kernel, l’OV5640 en
est un bon exemple. Il s’agit d’un capteur vidéo développé par Omnivision qui se trouve
être ressemblant à l’imx219 au niveau du protocole de communication avec la carte mère.
Le pilote déjà implémenté servira alors de base pour la création d’un driver imx219.

\subsection{Organisation d'un driver MIPI/CSI}

On dénombre 3 sections principale dans ce genre de pilote :
La première contient en majeure partie du code lié au fonctionnement du composant. On y
retrouve des structures contenant les registres à configurer, les différents modes
d’acquisitions d’image tel que la résolution, la gestion des régulateurs, les signaux de
démarrage et d’extinction, et d’autres caractéristiques. Cette section étant spécifique au
périphérique, c’est ici que seront modifiés un grand nombre de fonction. En effet les
registres et les séquences d’initialisation sont différentes entre les deux composants, il est
donc nécessaire de les adapter.
La deuxième est constituée de fonctions de contrôle et d’initialisation propres à V4L2.
Elles permettent d’enregistrer et d’utiliser l’imx219 en tant qu’appareil V4L2.
La dernière partie est relative à la gestion de la communication i2c, les fonctions
classiques d’un périphérique i2c sont présentes tel que ov5640\_probe, ov5640\_remove,
ov5640\_init et ov5640\_clean sans oublier la structure reliant les fonctions au système
ov5640\_i2c\_driver.

\subsubsection{Intégration au device tree}

Le device tree est un sous-ensemble composé de plusieurs fichiers configurant toutes les
liaisons entre le matériel et le logiciel. Il permet d’utiliser des drivers conçus pour linux. On
le retrouve dans la partition boot qui est accessible en lecture/écriture, cela permet de le
modifier facilement. Nous pouvons donc tester les drivers de façon rapide et efficace.

La configuration du driver imx219 :

\textbf{extrait de openrexbasic.dts}

\begin{lstlisting}
    &i2c2 {
/* Raspberry Pi camera rev 2.1 */
camera: imx219_mipi@64{
compatible = "sony,imx219";reg = <0x64>;
clocks = <&clks IMX6QDL_CLK_DUMMY>;
clock-names = "csi_mclk";
DOVDD-supply = <&reg_1p8v>;
AVDD-supply = <&reg_2p8v>;
DVDD-supply = <&reg_1p5v>;
pwn-gpios = <&gpio7 6 GPIO_ACTIVE_HIGH>;
csi_id = <1>;
mclk = <24000000>;
mclk_source = <0>;
pinctrl-names = "default";
pinctrl-0 = <&pinctrl_imx219>;
};
\end{lstlisting}

\textbf{extrait de imx6qdl-openrex.dtsi}

\begin{lstlisting}
    &mipi_csi 
    {
        ipu_id = <0>;
        csi_id = <1>;
        v_channel = <1>;
        lanes = <2>;
        mipi_dphy_clk = <0x28>;
        status = "okay";
    };

    pinctrl_imx219: imx219_grp
    {
        fsl,pins = <MX6QDL_PAD_SD3_DAT2__GPIO7_IO06 0x00017059>;
    };
\end{lstlisting}

l’extrait ci-dessus représente la déclaration du driver imx219 dans le \\
device tree.

\begin{itemize}
    \item[-]compatible = “sony,imx219” : correspond au nom associé
    dans la structure imx219\_i2c\_driver sous la variable .name
    \item[-]reg =<0x64> : correspond à l’adresse i2c du périphérique
    \item[-] clocks = <\&clks IMX6QDL\_CLK\_DUMMY> : lien vers l'horloge à utiliser
    \item[-] clock-names = "csi\_mclk" : nom de l’horloge
    \item[-] pwn-gpios = <\&gpio76GPIO\_ACTIVE\_HIGH> : gpio qui contrôle l’allumage
    \item[-] csi\_id = <1> : identifiant vers la structure CSI
    \item[-] mclk = <24000000> : fréquence de l’horloge
    \item[-] mclk\_source = <0> : source de l’horloge en accord avec la structure v4l2\_cap\_1
    \item[-] pinctrl-0 = <\&pinctrl\_imx219> : lien vers la déclaration des gpios
\end{itemize}

\subsection{Validation du driver}

Au démarrage du driver, la première fonction exécutée est imx219\_probe. \\
Comme son nom l’indique, elle a pour rôle de sonder certaines parties du composant comme la broche
de démarrage (pwn-gpio) , l’horloge (mclk), le csi\_id etc... En d’autres termes, elle vérifie
si les paramètres donnés par le device tree sont cohérents. Pour s’assurer de la bonne
communication avec le composant, Sony a prévu un registre contenant l’ID de la caméra.
Une des vérifications de débogage de la fonction probe est de lire ce registre et de tester
sa valeur. Or lors du chargement du driver dans le kernel une erreur relative a l’ID
apparaissait. \medskip 

Afin de pouvoir relier les erreurs logicielles émises par le driver à des erreurs de
manipulation de l’imx219 interprétables la datasheet, nous avons monitoré la
communication I2C entre l’Openrex et l’imx219.

\subsubsection{Lecture du bus I2C}

Dans un premier temps, nous avons réalisé plusieurs lectures à l’oscilloscope, puis de
manière bien plus efficace nous avons utilisé un analyseur logique relié à un PC. Grâce à
ce montage, nous avons pu lire le bus i2c tout au long de la phase de boot, mais aussi,
nous avons pu interpréter les trames logiciellement en hexadécimal via l’application
“Saleae Logic”. \medskip

\begin{figure}[!htb]
    \centering
    \includegraphics[trim={0cm 0cm 0cm 0cm},clip,scale=0.35]{Figures/trame.png}
    \decoRule
    \caption{Trame I2C} \label{fig:trame}
\end{figure} 

Sur le bus I2C, on a pu dénombrer 6 périphériques adressés 0x14, 0x15, 0xC8, 0xC9,
0x90 et 0x91. Nous nous sommes intéressé principalement à 0xC8 et 0xC9
vraisemblablement le SOM caméra et son interlocuteur (Openrex). On trouve dans les
communications qui lui sont adressées ce premier échange. dont nous ne sommes pas
sûr de l’interprétation.

\begin{itemize}
    \item[-] Setup Read to [0xC9] + ACK
    \item[-] 0x04 + NAK
    \item[-] Setup Write to [0xC8] + ACK
    \item[-] 0x00 + ACK
    \item[-] 0x01 + ACK
    \item[-] Setup Read to [0xC9] + NAK
\end{itemize}

Puis apparait le début de la structure de communication écrite dans le driver. Elle permet
de déverrouiller les configurations du fabricant mais n’est pas émise dans sa totalité et
s'interrompt au 3ème registre.

\begin{itemize}
    \item[-]Setup Write to [0xC8] + ACK
    \item[-]0x30 + ACK
    \item[-]0xEB + ACK
    \item[-]0x05 + ACK
    \item[-]Setup Write to [0xC8] + ACK
    \item[-]0x30 + ACK
    \item[-]0xEB + ACK
    \item[-]0x0C + ACK
    \item[-]Setup Write to [0xC8] + ACK
    \item[-]0x30 + ACK
    \item[-]0x0A + ACK
    \item[-]0xFF + ACK
    \item[-]Setup Write to [0xC8] + NAK
\end{itemize}
% Chapter 4

\chapter{Conclusion} % Main chapter title

\label{Chapter4} % For referencing the chapter elsewhere, use \ref{Chapter4} 

%----------------------------------------------------------------------------------------

\section{Bilan technique (AA,ML)}

\begin{table}[htp]
    \centering
    \begin{tabular}{|p{0.6\textwidth}|p{0.15\textwidth}|}
        \hline
        Mise en place d'un environnement de compilation & Terminé \\ \hline
        Développement d'un OS bootable sur l'OpenRex & Terminé \\ \hline
        Driver imx219 & En cours \\ \hline
    \end{tabular}
    \caption{Conclusion du projet} \label{tab:conclusion} 
\end{table}

Au cours du projet, nous somme tombés sur plusieurs impasses qui nous ont permis
d’apprendre les différentes façons de compiler un driver avec Yocto. \medskip

De part notre faible connaissance en driver Linux, nous ne pensions pas avoir les
capacités techniques requises pour développer nous-même un driver. Notre travail c’est
donc orienté vers trois drivers imx219 existants. Le BSP Openrex étant compatible avec le
kernel 3,14 et 4,1 nous étions limités en ressources. Deux des drivers n’était pas
compatible avec notre kernel, nous avons alors cherché à déterminer quelles
bibliothèques étaient responsable de cette incompatibilité. Malheureusement, les versions
étaient trop éloignées pour imaginer patcher toutes les bibliothèques utiles au
fonctionnement des drivers.\medskip

Une dernière solution était de rendre compatible le driver avec notre kernel, après l’avoir
rendu compatible avec notre device tree nous avions un segmentation fault lors du chargement
du kernel. Simultanément nous portions le bsp de l’openrex sur un kernel 4.14 qui n’a pas pu être
testé suite à une erreur survenue avec le bootloader.\medskip

Face à ces multiples échecs et un échange avec le client, nous commencions à rédiger
notre propre driver en se basant sur ceux déjà inclus dans le kernel. Le driver est
maintenant compilé et configuré par le device tree cependant il nous est impossible de lire
ou d’écrire dans un registre de la carte. Notre travaille s’achève donc sur ce point.\medskip

Techniquement nous avons acquis un bagage de connaissances concernant l’usage des
couches applicatives v4l2 nécessaire à la capture d’images de l’environnement Yocto. À
l’issue de ce rapport, nous pouvons nous concentrer sur le développement du code en
langage C.

\section{Bilan de suivi de projet (AA,ML)}

Dès le commencement du projet, nous somme partis en méthode agile, notre groupe de
travail a su s’auto-organiser et a perduré jusqu’à la fin du temps imparti. Nous avons
rapidement et facilement réussi à répartir le travail en fonction des compétences de
chacun, des obstacles matériels et logistiques rencontrés. \medskip

Malgré les différences de niveaux initials dûes au passif technologique de chacun, chaque individu
à apporter son utilité. En revanche, si le côté, communication et adaptabilité de la méthode agile
est respecté, le lien avec le client quant à lui a été négligé. \medskip

C’est en partie dû à la séparation physique du scrum-master et du groupe puis au manque d’outils
mise en place pour faciliter ce rapprochement. Une communication plus efficace avec l’équipe de
Thales nous aurait évité par exemple de prolonger trop longtemps la piste des drivers existants.

\section{Conclusion (AA,ML)}

Nous n’avons pas pu répondre complètement à la demande de Thales, qui est
actuellement entrain de développer le driver avec des résultats encouragent. Face au
obstacle notre groupe a toujours cherché à progresser en allant de plus en plus loin dans
le raisonnement technique. Bien quinachevé, cette expérience reste une des plus
enrichissantes de notre année. Étant soucieux d’apporter notre pierre à l’édifice nous
laissons avec ce rapport un environnement de développement Yocto optimisé pour
compiler un OS compatible avec l’Openrex et un guide d’utilisation et de développement
en annexe. 
%% Chapter 5

\chapter{État de l'art de la récupération du flux vidéo} % Main chapter title

\label{Chapter5} % For referencing the chapter elsewhere, use \ref{Chapter5} 

%----------------------------------------------------------------------------------------

\section{Webcam USB}

Afin de mieux comprendre l’utilisation de Video4Linux nous avons effectué plus de
recherches sur celui-ci et nous sommes tombés sur un blog expliquant la récupération d’un
flux vidéo d’une webcam USB avec V4L2 sur imx6. Nous avons donc essayé de capturer le flux
de notre webcam USB dans un premier temps, une fois ceci fonctionnel nous passerons sur le
flux vidéo de la Raspi Cam v2. Nous avons récupéré une webcam USB afin de tester les
différentes commandes proposées par le blog et essayer de capturer le retour vidéo.

Dans un premier temps il est nécessaire de récupérer sur quel port est connecté la
webcam USB, pour cela on utilise la commande : \textbf{lsusb -t}

\begin{figure}[th]
    \centering
    \includegraphics[width=1\linewidth,trim={0cm 0,6cm 0cm 12,5cm},clip]{webcam1.png}
    \decoRule
    \caption{Port USB de la webcam USB}  \label{fig:webcam1}   
\end{figure}

D’après la figure ci-dessus, nous en déduisons que la webcam USB est sur le bus
I2C n°1 et l’identifiant du port 1.2.

\begin{figure}[th]
    \centering
    \includegraphics[width=1\linewidth,trim={0cm 0,6cm 0cm 14,5cm},clip]{webcam2.png}
    \decoRule
    \caption{Vérification du port et identifiant de la webcam USB}  \label{fig:webcam2}   
\end{figure}

Ensuite il faut écrire le numéro du bus et l’identifiant du port dans les fichiers
bind et unbind. Le fichier bind permette de  « lier » un appareil USB à son driver et
ainsi le rendre visible pour le système tandis que le fichier unbind lui détache le driver
de son appareil USB pour le « cacher » du système.

\begin{figure}[th]
    \centering
    \includegraphics[scale=0.5,trim={0cm 11,5cm 0cm 1,7cm},clip]{webcam3.png}
    \decoRule
    \caption{Liaison entre la webcam USB et son driver}  \label{fig:webcam3}   
\end{figure}

On s’aperçoit bien que la webcam USB est bien reconnu par notre système puisquue
celui-ci nous affiche la référence de la webcam USB comme le montre l’illustration ci-dessus.

\begin{figure}[th]
    \centering
    \includegraphics[scale=0.4,trim={0cm 26,1cm 0cm 1,7cm},clip]{webcam4.png}
    \decoRule
    \caption{Capture du flux vidéo de la webcam USB}  \label{fig:webcam4}   
\end{figure}

La commande \textbf{\# gst-launch-1.0 v4l2src device="/dev/video0" ! video/x-raw,w \\
idth=640,height=480 ! autovideosink} permet de démarrer un flux vidéo sur notre webcam USB.
Nous en sommes sûr puisque lorsque nous lançons la commande celle-ci se fige à « New clock :
GstSystemClock » , ça se fige exactement au même endroit que si nous effectuons cette commande
sur notre propre ordinateur et notre webcam intégré. De plus une petite LED verte s’allume
au moment de l’envoi de cette commande.

\begin{figure}[th]
    \centering
    \includegraphics[scale=0.2,trim={10cm 2cm 4cm 9,5cm},clip]{webcam5.png}
    \decoRule
    \caption{LED verte allumée lors de l'envoi de la commande}  \label{fig:webcam5}   
\end{figure}

\section{Problèmes}

Notre principal problème pour l’instant est de récupérer ce flux vidéo puisqu’il
est impossible de l’afficher directement dans le terminal, pour tester cela nous
essayons de l’afficher sur nos ordinateurs par le réseau WiFi. 

\section{Point d'amélioration}

Dans un premier temps il est nécessaire de mettre l’image à jour pour que l’openrex
basic se connecte à un réseau WiFi, qui sera le même que nos ordinateurs, et dans un
second temps de récupérer ce flux vidéo par le réseau WiFi sur nos ordinateurs en UDP.


Finalement une fois que nous aurons réussi à afficher le flux vidéo de la webcam USB
sur nos ordinateurs nous essayerons d’effectuer le même processus mais pour la Raspi Cam v2,
qui est se pilote par le CSI et non l’USB.

\section{Sources}

\href{http://jas-hacks.blogspot.fr/2014/04/imx6-gstreamer-imx-and-usb-webcam.html}{http://jas-hacks.blogspot.fr/2014/04/imx6-gstreamer-imx-and-usb-webcam.html} 

%----------------------------------------------------------------------------------------
%	THESIS CONTENT - APPENDICES
%----------------------------------------------------------------------------------------

%\appendix % Cue to tell LaTeX that the following "chapters" are Appendices

% Include the appendices of the thesis as separate files from the Appendices folder
% Uncomment the lines as you write the Appendices

%\include{Appendices/AppendixA}
%\include{Appendices/AppendixB}
%\include{Appendices/AppendixC}

%----------------------------------------------------------------------------------------
%	BIBLIOGRAPHY
%----------------------------------------------------------------------------------------

%\printbibliography[heading=bibintoc]

%----------------------------------------------------------------------------------------

\end{document}  
