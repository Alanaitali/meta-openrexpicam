% Chapter 4

\chapter{Conclusion} % Main chapter title

\label{Chapter4} % For referencing the chapter elsewhere, use \ref{Chapter4} 

%----------------------------------------------------------------------------------------

\section{Bilan technique}

\begin{table}[htp]
    \centering
    \begin{tabular}{|p{0.6\textwidth}|p{0.15\textwidth}|}
        \hline
        Mise en place d'un environnement de compilation & Terminé \\ \hline
        Développement d'un OS bootable sur l'OpenRex & Terminé \\ \hline
        Driver imx219 & En cours \\ \hline
    \end{tabular}
    \caption{Conclusion du projet} \label{tab:conclusion} 
\end{table}

Au cours du projet, nous somme tombés sur plusieurs impasses qui nous ont permis
d’apprendre les différentes façons de compiler un driver avec Yocto. \medskip

De part notre faible connaissance en driver Linux, nous ne pensions pas avoir les
capacités techniques requises pour développer nous-même un driver. Notre travail c’est
donc orienté vers trois drivers imx219 existants. Le BSP Openrex étant compatible avec le
kernel 3,14 et 4,1 nous étions limités en ressources. Deux des drivers n’était pas
compatible avec notre kernel, nous avons alors cherché à déterminer quelles
bibliothèques étaient responsable de cette incompatibilité. Malheureusement, les versions
étaient trop éloignées pour imaginer patcher toutes les bibliothèques utiles au
fonctionnement des drivers.\medskip

Une dernière solution était de rendre compatible le driver avec notre kernel, après l’avoir
rendu compatible avec notre device tree nous avions un segmentation fault lors du chargement
du kernel. Simultanément nous portions le bsp de l’openrex sur un kernel 4.14 qui n’a pas pu être
testé suite à une erreur survenue avec le bootloader.\medskip

Face à ces multiples échecs et un échange avec le client, nous commencions à rédiger
notre propre driver en se basant sur ceux déjà inclus dans le kernel. Le driver est
maintenant compilé et configuré par le device tree cependant il nous est impossible de lire
ou d’écrire dans un registre de la carte. Notre travaille s’achève donc sur ce point.\medskip

Techniquement nous avons acquis un bagage de connaissances concernant l’usage des
couches applicatives v4l2 nécessaire à la capture d’images de l’environnement Yocto. À
l’issue de ce rapport, nous pouvons nous concentrer sur le développement du code en
langage C.

\section{Bilan de suivi de projet}

Dès le commencement du projet, nous somme partis en méthode agile, notre groupe de
travail a su s’auto-organiser et a perduré jusqu’à la fin du temps imparti. Nous avons
rapidement et facilement réussi à répartir le travail en fonction des compétences de
chacun, des obstacles matériels et logistiques rencontrés. \medskip

Malgré les différences de niveaux initials dûes au passif technologique de chacun, chaque individu
à apporter son utilité. En revanche, si le côté, communication et adaptabilité de la méthode agile
est respecté, le lien avec le client quant à lui a été négligé. \medskip

C’est en partie dû à la séparation physique du scrum-master et du groupe puis au manque d’outils
mise en place pour faciliter ce rapprochement. Une communication plus efficace avec l’équipe de
Thales nous aurait évité par exemple de prolonger trop longtemps la piste des drivers existants.

\section{Conclusion}

Nous n’avons pas pu répondre complètement à la demande de Thales, qui est
actuellement entrain de développer le driver avec des résultats encouragent. Face au
obstacle notre groupe a toujours cherché à progresser en allant de plus en plus loin dans
le raisonnement technique. Bien quinachevé, cette expérience reste une des plus
enrichissantes de notre année. Étant soucieux d’apporter notre pierre à l’édifice nous
laissons avec ce rapport un environnement de développement Yocto optimisé pour
compiler un OS compatible avec l’Openrex et un guide d’utilisation et de développement
en annexe.