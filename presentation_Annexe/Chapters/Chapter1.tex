% Chapter 1

\chapter{Introduction} % Main chapter title

\label{Chapter1} % For referencing the chapter elsewhere, use \ref{Chapter1} 

%----------------------------------------------------------------------------------------

\color{red}

Ce rapport est un rapport d’activité sur le travail effectué sur le rojet proposé par Thales. Ce
rapport a été écrit par les 4 élèves participants au projet. Pour un souci de notation il nous a été
demandé de signaler quelle partie a été traité par quel élève. Afin de ne pas rendre le rapport
illisible en expliquant à chaque fois qui c’est chargé de faire telle ou telle partie nous avons ajouté
des annotations en bas de page.

\color{black}

\section{Projet professionnel}

Les technologies d’affichage à tête haute ou HUD (head up display) apparaissent de plus en plus dans le
commerce grand public (voitures, lunettes, chirurgies…). Il y a quelques années Thales a développé un
casque à visée tête haute pour les pilotes militaires. Ce système permet de projeter des informations
sur la vitre du casque du pilote sans gêner la vue réelle en arrière-plan. Cela évite aux avions de 
faire un premier passage pour repérer visuellement la cible avant de passer à l’action. À présent, la
cible est balisée à vue dès l’arrivée de l’avion. Récemment Thales s’est lancé en interne à adapter 
ce produit pour des applications civiles. \medskip

Ce projet, appelé LUCY, a passé l’étape du proof of concept (POC). Aujourd’hui les ingénieurs
travaillent sur son amélioration en minimum viable product (MVP) pour commencer à équiper le système
de potentiels clients et à déterminer plus précisément les besoins de ceux-ci. 

\section{Projet étudiant}

Notre travail au sein du projet professionnel consiste à améliorer la méthode permettant de détecter et suivre
d’orientatier la tête du pilote par rapport au cockpit. En particulier lors de la phase de
calibration, où il faut ajuster la ligne d’horizon sur la visière du pilote. Pour assurer la
compatibilité des drivers, la preuve de faisabilité (proof of concept POC) était initialement
constituée d’une caméra Raspberry-Pi (v2) contrôlée par une carte Raspberry Pi. Au passage au MVP
les responsables du projet ont pris la décision de changer de carte tout en gardant la caméra
Raspberry Pi. Notre rôle dans le projet est de préparer la carte Openrex pour l’acquisition d’images
et de flux vidéo en mettant en place un driver compatible. 

\cfoot{pied de page centré}

\subsection{Cadre du projet étudiant}

Le projet étudiant a pour but de permettre à la carte mère (single board computer, SBC) Openrex-basic
avec microcontrôleur (system on chip, SOC) IMX6S, de contrôler le module (system on module, SOM)
“Raspberry Pi Camera v2”, de la fondation Raspberry Pi, pour réaliser des captures d’image et de flux
vidéo. \medskip

Afin que le code applicatif existant sur la raspberry puisse être porté sur l’Openrex, on proposera
donc un système d’exploitation (operating system, OS) avec la même version de kernel (4.1) que celle
mise en place sur le POC.\medskip

Pour écourter la normalisation du produit, le système ne devra en aucun cas être en interaction avec
les équipements avioniques, déjà présents sur l’appareil.
